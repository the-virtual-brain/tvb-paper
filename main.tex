%!TEX TS-program = pdflatex                                                    %
%!TEX encoding = UTF8                                                          %
%!TEX spellcheck = en-US                                                       %
%------------------------------------------------------------------------------%
% to compile use "latexmk --pdf main.tex"                                      %
%------------------------------------------------------------------------------%
% to count words 
% "pdftotext main_nofigs_nocaptions.pdf - | egrep -e '\w\w\w+' | iconv -f ISO-8859-15 -t UTF-8 | wc -w"
% -----------------------------------------------------------------------------%

\documentclass{bioinfo}
\usepackage{url}
\usepackage[english]{babel}
\usepackage[utf8]{inputenc}
\usepackage[T1]{fontenc}
%\usepackage[pdftex]{graphicx} 
%\usepackage{graphics}
%\usepackage{hyperref}
\usepackage{float}
\floatplacement{figure}{H}
\usepackage{booktabs}     % nice tables
\usepackage{tabularx}      % even nicer tabular environments 
\usepackage{amsmath}
\usepackage{amsfonts}
\usepackage{amssymb}
%\usepackage{multicol}
\usepackage{listings}
\usepackage{tikz,times}
\usepackage{courier}
\usetikzlibrary{shapes,arrows}
\usetikzlibrary{arrows,positioning}
\usepackage{xcolor}
\usepackage[font=bf]{subfig}
\usepackage[finalnew]{trackchanges} % track changes from authors.
%\usepackage{sectsty}
%\sectionfont{\normalsize\bfseries}
%\usepackage[labelfont=bf]{caption}
%\usepackage{endfloat} %place figures at end of document
%------------------------------------------------------------------------------%
%\captionsetup{
%%format = hang,                % caption format
%labelformat = simple,          % caption label : name and number
%labelsep = period,             % separation between label and text
%textformat = simple,           % caption text as it is
%justification = justified,     % caption text justified
%singlelinecheck = true,        % for single line caption text is centered
%font = {up,singlespacing},     % defines caption (label & text) font
%labelfont = {bf,footnotesize}, % NOTE: tiny size is not working
%textfont = footnotesize,
%%width = \textwidth,           % define width of the caption text
%skip = 1ex,                    % skip the space between float and caption
%listformat = simple,           % in the list of floats, label + caption
%}

%------------------------------------------------------------------------------%
%\hypersetup{
%    bookmarks=true,         % show bookmarks bar?
%    unicode=false,          % non-Latin characters in Acrobat’s bookmarks
%    pdftoolbar=true,        % show Acrobat’s toolbar?
%    pdfmenubar=true,        % show Acrobat’s menu?
%    pdffitwindow=false,     % window fit to page when opened
%    pdfstartview={FitH},    % fits the width of the page to the window
%    pdftitle={TheVirtualBain},    % title
%    pdfauthor={PSL},        % author
%    pdfsubject={ProposedArticle},   % subject of the document
%    pdfcreator={paupau},    % creator of the document
%    pdfnewwindow=true,      % links in new window
%    colorlinks=true,       % false: boxed links; true: colored links
%    linkcolor=red,          % color of internal links (change box color with linkbordercolor)
%    citecolor=blue,        % color of links to bibliography
%    filecolor=magenta,      % color of file links
%    urlcolor=blue           % color of external links
%}
%-----------------------------------------------------------------------
%\usepackage{subcaption}

%%%%%%%%%%%%%%%%%%%%%%%%%%%%%%%%%%%%%%%%%%%%%%%%%%%%%%%%%%%%%%%%%%%%%%%%%%%%%%%%
%%                             New and renew commands                         %%
%%%%%%%%%%%%%%%%%%%%%%%%%%%%%%%%%%%%%%%%%%%%%%%%%%%%%%%%%%%%%%%%%%%%%%%%%%%%%%%%

% define authors
\addeditor{PSL}
\addeditor{MW}
\addeditor{VJ} 
\addeditor{SAK}

\renewcommand{\lstlistingname}{Code}
\renewcommand{\thesubfigure}{\Alph{subfigure}}
\newcommand{\inputTikZ}[2]{%
\scalebox{#1}{\input{#2}}}
\newcommand*{\h}{\hspace{5pt}}   % for indentation
\newcommand*{\hh}{\h\h}          % double indentation
\newcommand{\TVB}{\textit{TheVirtualBrain }}
\newcommand*{\tvbmodule}[1]{{\textsc{#1}}}          % scientific library modules
\newcommand*{\tvbdatatype}[1]{\textbf{\emph{#1}}}   % datatypes in "datatypes"
\newcommand*{\tvbclass}[1]{{\ttfamily\emph{#1}}}    % classes either in simulator mods or datatypes
\newcommand*{\tvbmethod}[1]{{\textsf{#1}}}          % methods
\newcommand*{\tvbattribute}[1]{{\ttfamily{#1}}}     % attributes
\newcommand*{\tvbtrait}[1]{{\ttfamily{#1}}}         % traited types

% please please trackchanges ignore these commands and make my life easy.
\tcregister{\tvbmodule}{1} 
\tcregister{\tvbdatatype}{1} 
\tcregister{\tvbclass}{1} 
\tcregister{\tvbmethod}{1} 
\tcregister{\tvbattribute}{1} 
\tcregister{\tvbtrait}{1} 
\tcregister{\TVB}{0}



%%%%%%%%%%%%%%%%%%%%%%%%%%%%%%%%%%%%%%%%%%%%%%%%%%%%%%%%%%%%%%%%%%%%%%%%%%%%%%%%
%%                            Colors and graphics                             %%
%%%%%%%%%%%%%%%%%%%%%%%%%%%%%%%%%%%%%%%%%%%%%%%%%%%%%%%%%%%%%%%%%%%%%%%%%%%%%%%%
\definecolor{palegreen}{HTML}{DAFFDA}
\definecolor{lightgray}{rgb}{0.15,0.15,0.15}
\definecolor{orange}{HTML}{FF7300}
\DeclareGraphicsExtensions{.jpg,.pdf,.png,.tiff}%,.mps,.bmp
\graphicspath{{figures/}}
 
%##--------------------------------------------------------------------------##%
%##                               START HERE                                 ##%
%##--------------------------------------------------------------------------##%
\copyrightyear{}
\pubyear{}

\begin{document}
\lstset{language=Python, 
        caption=b, 
        breaklines=true, 
        basicstyle=\bf\tiny\ttfamily, 
        stringstyle=\color{magenta}
        } 
\firstpage{1}

%%  Authorship and Title
\title[TVB]{The Virtual Brain: a simulator of primate brain network dynamics}
\author[Sanz Leon {et~al}]{Paula Sanz Leon\,$^{1,*}$,
        Stuart A. Knock\,$^{2}$, 
        M. Marmaduke Woodman\,$^{1}$,
        Lia Domide\,$^{3}$, 
        Jochen Mersmann\,$^{4}$,
        Anthony R. McIntosh \,$^{5}$ and  
        Viktor Jirsa\,$^{1}$\footnote{to whom correspondence should be addressed: paula.sanz-leon@univ-amu.fr, 
        viktor.jirsa@univ-amu.fr}}

\address{$^{1}$ Institut de Neurosciences des Syst{\`e}mes, 27, Bd. Jean Moulin, 13005, Marseille, France.\\
         $^{2}$ BrainModes Group, Department of Neurology, Charit{\'e} University of Medicine, Berlin, Germany.\\
         $^{3}$ Codemart, 13, Petofi Sandor, 400610, Cluj-Napoca, Romania.\\
         $^{4}$ CodeBox GmbH, Hugo Eckener Str. 7, 70184 Stuttgart, Germany.\\
         $^{5}$ Rotman Research Institute at Baycrest, Toronto, M6A 2E1, Ontario, Canada\\
        }

\history{}

\editor{}

\maketitle

%##--------------------------------------------------------------------------##%
%##                               ABSTRACT                                   ##%
%##--------------------------------------------------------------------------##%


\begin{abstract} \section{} % abstract < 2000 characters including spaces. 
We present TheVirtualBrain (TVB), a neuroinformatics platform for full brain
network simulations using biologically realistic connectivity. This simulation
environment enables the model-based inference of neurophysiological mechanisms
across different brain scales that underlie the generation of macroscopic
neuroimaging signals including functional MRI (fMRI), EEG and MEG. Researchers
from different backgrounds can benefit from an integrative software platform
including a supporting framework for data management (generation,
organization, storage, integration and sharing) and a simulation core written
in Python. TVB allows the reproduction and evaluation of personalized
configurations of the brain by using individual subject data. This
personalization facilitates an exploration of the consequences of pathological
changes in the system, permitting to investigate potential ways to counteract
such unfavorable processes. The architecture of TVB supports interaction with
MATLAB packages, for example, the well known Brain Connectivity Toolbox. TVB
can be used in a client-server configuration, such that it can be remotely
accessed through the Internet thanks to its \change[PSL]{web based}{web-based}
HTML5, JS and WebGL graphical user interface. TVB is also
accessible as a standalone cross-platform Python library and application, and
users can interact with the scientific core through the scripting interface
IDLE, enabling easy modeling, development and debugging of the scientific
kernel. This second interface makes TVB extensible by combining it with other
libraries and modules developed by the Python scientific community. 
In this article, we
describe the theoretical background and foundations that led to the
development of TVB, the architecture and features of its major software
components as well as potential neuroscience applications.
    
\section{Keywords:} connectivity, connectome, neural mass, neural field, 
time delays, full-brain network model, python, virtual brain, large-scale 
simulation,  web platform, GPUs
\end{abstract}

%##--------------------------------------------------------------------------##%
%##                               INTRO                                      ##%
%##--------------------------------------------------------------------------##%

\section{Introduction}
    
    % micro vs macro
    Brain function is thought to emerge from the interaction of large numbers
    of neurons, under the spatial and temporal constraints of brain structure
    and cognitive demands. Contemporary network simulations mainly focus on
    the microscopic and mesoscopic level (neural networks and neural masses
    representing a particular cortical region), adding detailed biophysical
    information at these levels of description while too often losing
    perspective on the global dynamics of the brain. On the other hand, the
    degree of  assessment of global cortical dynamics,  across imaging
    modalities, in human patients and research subjects has  increased
    significantly in the last few decades. In particular, cognitive and
    clinical neuroscience employs imaging methods of macroscopic brain
    activity such as intracerebral measurements, stereotactic Encephalography
    (sEEG) \citep{Ellenrieder_2012}, Electroencephalography (EEG) 
    \citep{Nunez_1981book, Nunez_1995book, Niedermeyer_2005}, 
    Magnetoencephalography (MEG) \citep{Haemaelaeinen_1992, Haemaelaeinen_1993, 
    Mosher_1999} and
    functional Magnetic  Resonance Imaging (fMRI) \citep{Ogawa_1993, Ogawa_1998, 
    Logothetis_2001} to assess brain dynamics and
    evaluate diagnostic  and therapeutic strategies. Hence, there is a strong
    motivation to develop  an efficient, flexible, neuroinformatics platform
    on this macroscopic level of brain organization for reproducing and
    probing the broad repertoire of  brain dynamics, enabling quick data
    analysis and visualization of the  results.
    
    TVB is our response to this need. On the one hand, it
    provides a general infrastructure to support multiple users handling
    various kinds of empirical and simulated data, as well as tools for
    visualizing and analyzing that data, either via internal mechanisms or by
    interacting with other computational systems such as MATLAB. At the same
    time it provides a simulation toolkit to support a top-down modeling
    approach to whole brain dynamics, where the underlying network is defined
    by its structural large-scale connectivity and mesoscopic models that govern
    the nodes' intrinsic dynamics. The interaction with the dynamics of all
    other network nodes happens through the connectivity matrix via specific
    connection weights and time delays, where the latter make a significant
    contribution to the biological realism of the temporal structure of
    dynamics.
    
    Historically, \citet{Jirsa_2002} first demonstrated neural field modeling on
    a spherical brain hemisphere employing EEG and MEG forward solutions to
    obtain simulation imaging signals. In this work, homogeneous
    (translationally invariant) connectivity was implemented along the lines of
    \citet{Jirsa_1996,Jirsa_1997, Bojak_2010} yielding a neural field equation,
    which has its roots in classic works
    \citep{Wilson_1972,Wilson_1973,Nunez_1974,Amari_1975, Amari_1977}. At that
    time more detailed large-scale connectivity of the full primate brain was
    unavailable, hence the homogeneous connectivity scaled up to the full brain
    was chosen as a first approximation \citep{Nunez_1974}. The approach proved
    successful for the study of certain phenomena as observed in large-scale
    brain systems including spontaneous  activity \citep{Breakspear_2003,
    Freyer_2011,Wright_1995,Rowe_2004,Robinson_2001, Robinson_2003}, evoked
    potentials \citep{Rennie_1999, Rennie_2002}, anesthesia \citep{Liley_2005},
    epilepsy \citep{Breakspear_2006}, sensorimotor coordination
    \citep{Jirsa_1996, Jirsa_1997} and more recently, plasticity \citep{Robinson_2011} (see \citet{Deco_2008a} and
    \citet{Jirsa_2004} for a review).
    
    Careful review of this literature though shows that these models mostly
    emphasize the temporal domain of brain organization, but leave the
    spatiotemporal organization untouched. This may be understood by the fact
    that the symmetry of the connectivity imposes constraints upon the range
    of the observable dynamics. This was pointed out early
    \change[PSL]{on}{by }\citet{Jirsa_2002}
    and a suggestion was made to integrate biologically realistic DTI based
    connectivity into full brain modeling efforts. Large scale brain  dynamics
    are basically expected to reflect the underlying anatomical connectivity
    between brain areas \citep{Deco_2011, Bullmore_2009}, even though
    structural connectivity is not the only constraint, but the transmission
    delays play an essential role in  shaping the brain network dynamics also
    \citep{Jirsa_2000, Ghosh_2008, Knock_2009, Jirsa_2010b}. Recent studies
    \citep{Pinotsis_2012} have systematically investigated the degree to which
    homogeneous approximations may serve to understand realistic connection
    topologies and have concluded that homogeneous approximations are more
    appropriate for mesoscopic descriptions of brain activity, but less well
    suited to address full brain network dynamics. All this
    \change[PSL]{underwrites}{underscores} the need to incorporate realistic
    connectivity into large scale brain network models. Thus the simulation
    side of TVB has evolved out of a research program seeking to identify and
    reproduce realistic  whole brain network dynamics, on the basis of
    empirical connectivity and neural field  models \citep{Jirsa_2010,
    Deco_2011}.

  \subsection{Modeling}

    In line with these previous studies, TVB incorporates a biologically
    realistic, large-scale connectivity of brain regions in the primate brain.
    Connectivity is mediated by long-range neural fiber tracts as identified by
    tractography based methods \citep{Hagmann_2008, Honey_2009, Bastiani_2012},
    or obtained from CoCoMac neuroinformatics database \citep{Koetter_2004,
    Koetter_2005, Bakker_2012}. In TVB, the  tract-lengths matrix of the
    demonstration connectivity dataset is symmetric due to the fiber detection
    techniques used to extract the information being insensitive to
    directionality. On the other hand, the weights matrix is asymmetric as it
    makes use of directional information contained in the tracer studies of the
    CoCoMac database. The symmetry (or lack thereof) is neither a modeling
    constraint nor an imposed restriction on the weights and tract-length matrices.
    Such specifics is inherent to the connectivity demonstration dataset included in
    the distribution packages of TVB. The general implementation for weights and
    tract lengths is a full $nodes\,\times\,nodes$ matrices without any symmetry restrictions.


    \remove{Two types of brain connectivity 
    are distinguished in TVB,
    that is region-based and surface-based connectivity. In the former case,
    the networks comprise discrete nodes and connectivity, in which each node
    models the neural population activity of a brain region and the
    connectivity is composed of interregional fibers.  The connectivity matrix
    defines the connection strengths and time delays via signal transmission
    between all network nodes. In the latter case, cortical and sub-cortical
    areas are modeled on a finer and more realistic scale, in which each point
    represents a neural population model. This approach allows a detailed
    spatial sampling, in particular of the cortical surface, resulting in a
    spatially continuous approximation of the neural activity as in neural
    field modeling (Deco et al., 2008). Thus the connectivity is composed of
    local intra-cortical and global inter-cortical fibers.}\add[PSL]{Two types
    of structural connectivity are distinguished in TVB, that is long- and
    short-range connectivity, given by the connectivity matrix and
    the folded cortical surface respectively. The connectivity matrix defines
    the connection strengths and time delays via finite signal transmission speed 
    between two regions of the brain. The cortical surface allows a more detailed
    spatial sampling resulting in a spatially continuous approximation of the
    neural activity as in neural field modeling} \citep{Deco_2008a, Coombes_2010, Bressloff_2012}.
    \add[PSL]{When using neural mass models, building the network upon the
    surface allows for the application of arbitrary local connectivity kernels
    which represent short-range intra-cortical connections. Additionally, networks
    themselves can be defined at two distinct spatial scales yielding two
    types of simulations (or brain network models): surface-based and 
    region-based. In the former case, cortical and sub-cortical areas are shaped 
    more realistically, each vertex of the surface is considered a node and is 
    modeled by a neural population model; several nodes belong to a specific
    brain region, and the edges of the network have a distance of the order of
    a few millimeters. The influence of delayed activity coming from other
    brain regions is added to the model via the long-range connectivity.
    In the latter case of nodes only per region, the connectome itself is used as
    a coarser representation of the brain network model. The networks comprise
    discrete nodes, each of them models the neural population activity of a
    brain region and the edges represent the long-range connectivity
    (interregional fibers) on the order of a few centimeters. Consequently, in 
    surface-based simulations both types of connectivity, short- and long-range, 
    coexist whereas in region-based simulations one level of  geometry is lost: 
    the short-range connectivity.}

    Neural field models have been developed over many years for their ability to
    capture the collective dynamics of relatively large areas of the brain in both
    analytically and computationally tractable forms \citep{Beurle_1956,
    Wilson_1972, Wilson_1973, Nunez_1974, Amari_1975, Amari_1977, Wright_1995,
    Jirsa_1996, Jirsa_1997, Robinson_1997, Jirsa_2002, Attay_2006, Bojak_2010}.
    Effectively neural field equations are tissue level models that describe the
    spatiotemporal evolution of coarse grained variables such as synaptic or firing
    rate activity in populations of neurons. Some of these models include explicit
    spatial terms while others are formulated without an explicit spatial component
    leaving open the possibility to apply effectively arbitrary local connectivity
    kernels. The lumped representation of the dynamics of a set of similar neurons
    via a common variable (e.g., mean firing rate and mean postsynaptic potential)
    is known as neural mass modeling \citep{Freeman_1975book, Freeman_1992,
    LopesdaSilva_1974}. Neural mass models accounting for parameter dispersion in
    the neuronal parameters include \citep{Assisi_2005, Stefanescu_2008,
    Stefanescu_2011, Jirsa_2010}.  Networks of neural masses, without an explicit
    spatial component within the mass but with the possibility to apply local
    connectivity kernels (e.g., Gaussian or Laplacian functions) between masses, can
    be used to approximate neural field models. Both neural field and neural mass
    modeling approaches embody the concept from statistical physics that macroscopic
    physical systems obey laws that are independent of the details of the
    microscopic constituents of which they are built \citep{book_Haken_1983}.  These
    and related ideas have been exploited in neurosciences \citep{Kelso_1995,
    Buzsaky_2006}.
        
    In TVB, our main interest lies in using the mesoscopic laws governing the
    behavior of neural populations and uncovering the laws driving the processes
    on the macroscopic brain network scale. The biophysical mechanisms
    available to microscopic single neuron approaches are absorbed in the mean
    field parameters on the mesoscopic scale and are not available for
    exploration other than through variation of the mean field parameters
    themselves. As a consequence, TVB represents a neuroinformatics tool that
    is designed to aid in the exploration of large-scale network mechanisms of
    brain functioning (see \citet{Ritter_2013} for an example of modeling with TVB). 

    \add{Furthermore, TVB's approach to multi-modal neuroimaging integration in
    conjunction with neural field modeling shares common features with the work of}
    \citet{Bojak_2010a, Bojak_2011} and \citet{Babajani_2010}\add{.     The crucial difference
    is that the structure upon which TVB has been      designed represents a
    generalized large-scale ``computational neural      model'' of the whole brain.
    The components of this large-scale model     have been separated as cleanly as
    possible, and a specific structure     has been defined for the individual
    components. This generic structure is intended to serve the purpose of
    restricting the form of models      enough to make direct comparison straight
    forward while still permitting a sufficiently large class of models to be
    expressed. Likewise, the     paradigms presented during the last few years in
    this line of research} \citep{Sotero_2007, Sotero_2008} \add{could potentially
    be reproduced, tested and compared in TVB. The mathematics underlying our model-based 
    approach have been partially described in various original articles}
    \citep{Deco_2011,Deco_2012} \add{and will be reviewed in more detail in future articles.}
        
  \subsection{Informatics}
    
    From an informatics perspective, a large-scale simulation project requires
    a well defined yet flexible workflow, i.e. adaptable according to the
    users profiles. A typical workflow in TVB involves managing project
    information, uploading data, setting up simulation parameters (model,
    integration scheme, output modality), launching simulations (in parallel
    if needed), analyzing and visualizing, and finally storing results and
    sharing output data.
    
    \change[PSL]{The web interface allows users without programming
    knowledge to access TVB to perform customized simulations using their
    patients' data.}{The web interface allows users without programming
    knowledge to access TVB to perform customized simulations (e.g. clinicians
    could use their patient's data obtained from DTI studies).} In addition,
    it enables them to gain a deeper understanding of the theoretical
    approaches behind the scenes. On the other hand, theoreticians can design
    their own models and get an idea of their biophysical realism, their
    potential physiological applications and implications. As both kinds of
    users may work within the same framework, the interplay of theory and
    experiment or application is accelerated. Additionally, users with
    stronger programming skills benefit from all the advantages provided by
    the Python programming language: easy-to-learn, easy-to-use, scriptable
    and with a large choice of scientific modules
    \citep{Oliphant_2006}\add[PSL]{.}
    
    % WHY PYTHON?
    TVB has been principally built in the Python programming language due to
    its unique combination of flexibility, existing libraries and the ease,
    with which code can be written, documented, and maintained by
    non-programmers.  The simulation core, originally developed in MATLAB, was
    ported to Python given its current significance in the numerical computing
    and neuroscience community and its already proven efficiency for
    implementing modeling tools \remove[PSL]{.}\citep{Spacek_2008}\add[PSL]{.}

    \add{Simulations benefit from vectorized numerical computations
    with NumPy arrays and are enhanced by the use of the \emph{numexpr} package.
    Although this allows rather efficient single simulations, the desire to
    systematically explore the parameter spaces of the neural dynamic models, and to
    compare many connectivity matrices, has lead to the implementation of code
    generation mechanisms for the majority of the simulator core -- producing C code
    for both     native CPU and also graphics processing units (GPU), leading to a
    significant speed up of parameter sweeps and parallel simulations (5x from
    Python to C, 40x from C to GPU). Such graphics units have become popular in
    scientific computing for their relatively low price and high computing power.
    Going forward, the GPU implementation of TVB will require testing and
    optimization before placing it in the hands of users. }

    
    This article intends to give a comprehensive but non-exhaustive
    description of TVB, from both technical and scientific points of view. It
    will describe the framework's architecture, the simulation core, and the
    user interfaces.  It will also provide two examples, using specific
    features of the simulator, extracted from the demo scripts which are
    currently available in TVB's distribution packages. \remove{The mathematics
    underlying the brain network models used in TVB is briefly described in
    various original articles and will be 
    reviewed in more detail elsewhere.}
    
    
%##--------------------------------------------------------------------------##%
%##                           SOFTWARE DESCRIPTION                           ##%
%##--------------------------------------------------------------------------##%
\begin{methods}

\section{TVB Architecture}
\note[PSL]{Smaller font is given by the Frontiers In LaTeX template}
    The architectural model of the system has two main
    components: the  scientific computing core and the supporting framework
    with its graphical user interface. Both software components communicate
    through an interface  represented by TVB-\tvbdatatype{Datatypes},
    \change{as}{which are} described in subsection \ref{subsection:tvb-datatypes}. 
    \add{In Fig. 1 TVB's architectural details are illustrated and explained in more
    depth.}
    
    \paragraph*{General aspects} TVB is designed for three main deployment
    configurations, according to the available hardware resources:  1) Stand
    Alone;  2) Client-Server or 3) Cluster. In the first, a local  workstation
    is assumed to have certain display, computing power and storage capacity
    resources. In the second, an instance of TVB is running on a server
    connected through a network link to client units, and thus accessible to a
    certain number of users. In this deployment model, simulations use the
    back-end server's computing power while visualization tasks use resources
    from the client machine. The third is similar to the client-server
    configuration, but with the additional advantage of parallelization
    support in the back-end. The cluster itself needs to be configured
    separately of TVB.
    
    Based on the usage scenarios and user's level of programming knowledge, two
    user profiles are represented: a graphical user (G-user) and scripting user
    (S-user).  We therefore provide the corresponding main interfaces based on
    this  classification: a graphical user interface (web) and a scripting
    interface (IDLE). S-users and G-users have different levels of control over
    different parts of the system.  The profile of S-users is thought to be that
    of scientific developers, that is, researchers who can elaborate complex
    modeling scenarios, add their own models or directly modify the source code
    to extend the scientific core of TVB, mostly working with the scientific
    modules.  They do, nevertheless, have the possibility to enable the database
    system. In contrast, G-users are relatively more constrained to the features
    available in the stable releases of TVB, since their profile corresponds
    more to that of researchers without a strong background in computational
    modeling. The distinction between these two profiles is mainly a
    categorization due to the design architecture of TVB. For instance, we could
    also think of other type of users who want to work with TVB's GUI and are
    comfortable with programming, and therefore they could potentially make
    modifications in the code and then see the effect of those when launching
    the application in a web browser.

    The development of TVB is managed under Agile techniques. In
    accord therewith, each task is considered as \emph{done}, after completing a validation
    procedure that includes: adding a corresponding automated unit-test, labeling the
    task as \emph{finished} from the team member assigned to implement
    the task and further tagging as \emph{closed} from a team member
    responsible for the module, which means a second level of testing. Before
    releasing stable packages, there is a period for manual testing, that is, a
    small group of selected users from different institutions check the main
    features and functionalities through both interfaces. The navigation and workflows scenarios
    through the web-based interface are evaluated by means of automated
    integration tests for web-applications running with Selenium
    (\url{http://docs.seleniumhq.org/}) and Apache-JMeter
    (\url{http://jmeter.apache.org/}) on top of a browser engine. 
    Special effort is being made to provide good code-coverage, including regression tests. Accordingly, the
    simulation engine of TVB has automated unit-tests, to guarantee the proper and
    coordinated functioning of all the modules, and simple programs
    (demonstration scripts), that permit to qualitatively evaluate 
    the scientific correctness of the results.
    
    The development version of TVB is currently
    hosted on a private cluster,  where we use the version control system
    \emph{svn} (subversion).  \add{Additionally, as any large collaborative
    open-source project, it is also available in a public repository, using
    the distributed version control system \emph{git}} \citep{Chacon_2009}
    \add{to make accessible the scientific core and to gather, manage and
    integrate contributions from the community.}  The
    distribution packages for TVB come with an extensive documentation,
    including: a \emph{User Guide}, explaining how to install TVB, set up
    models and run them; \emph{Tutorials}, \emph{Use Cases} and \emph{Script
    Demos}, guiding users to achieve predefined simulation scenarios; and a
    \emph{Developer Guide} and \emph{API reference}. \remove{There is an
    active users group of TVB hosted in Google Groups where users can ask
    questions, report issues and suggest improvements or new features.}  Table
    1 \add{provides the links to: the official TVB website,
    where distribution packages for Linux and Mac OS (32 and 64 bits) and
    Windows (32 bits) are available for download; the active users group of
    TVB hosted in Google Groups, where users can ask questions, report issues
    and suggest improvements or new features; and the public repository, where
    the source code of both the framework and scientific library (which
    contains the simulation engine) are available.}

  \note{New table providing links}
%--------------------------------TABLE--------------------------------------%%
  
  \begin{center}
    \begin{table}\label{tab:tvb_links}\tiny
      \caption{TVB links}
      \begin{tabularx}{0.5\textwidth}{l  l}
        \toprule
        TVB official website        &{\tiny{ 
        \url{http://www.thevirtualbrain.org}}}                        \\
        Distribution packages        & 
        {\tiny{\url{http://www.thevirtualbrain.org/register}}}        \\
        Public repository           & 
        {\tiny{\url{https://github.com/the-virtual-brain}}}           \\
        User group                  & 
        {\tiny{\url{https://groups.google.com/group/tvb-users/}}}     \\
        \bottomrule        
      \end{tabularx}
    \end{table}
  \end{center}  
%
%--------------------------------FIGURE--------------------------------------%%
%% Fig. 1       
%
%##--------------------------------------------------------------------------##%
%##                           INSTALL                                        ##%
%##--------------------------------------------------------------------------##%
%
    \paragraph*{Installation and System Requirements}
    
    When using the web \change{application}{interface}, users are
    \change{required}{recommended} to have a high  definition monitor (at
    least 1600 x 1000 pixels), a WebGL and WebSockets  compatible browser
    (latest versions of Mozilla Firefox, Apple Safari or  Google Chrome), and
    a WebGL-compatible graphics card, that supports OpenGL version 2.0 or
    higher \citep{opengl-redbook}.
    
    Regarding memory and storage capacity, for a stand alone installation a
    minimum of 8GB of RAM \add[PSL]{is recommended. For multi-users environments}   
    \change[PSL]{50GB}{5GB} of space per user is
    \change[PSL]{recommended}{suggested}. \add[PSL]{This is a storage quota 
    specified by an administrator to manage 
    the maximum hard disk space per user.} As for computing power one CPU core is needed
    for a simulation with a small number of nodes, while simulations with a
    large number of nodes, such as surface simulations, can make use of a few
    cores if they are available.  When the number of launched simulations is
    larger than the number of  available cores, a serialization is recommended
    (a serialization mechanism  is provided by the supporting framework
    through the web user interface by specifying the maximum of simultaneous
    jobs allowed).  In order to use the Brain Connectivity Toolbox
    \citep{Rubinov_2010},  MATLAB or Octave should be installed, activated and
    accessible for the  current user. \remove[PSL]{The Brain Connectivity
    Toolbox does not need to be  installed or enabled separately in any way,
    as TVB will temporarily append  its own copy of the toolbox to the
    MATLAB/Octave path.}

%##--------------------------------------------------------------------------##%
%##                           WEB - FRAMEWORK                                ##%
%##--------------------------------------------------------------------------##% 

  \subsection{TVB Framework}\label{subsec:TVBFramework}
    
    The supporting framework provides a database back-end, workflow management
    and a number of features to support collaborative work. \add{The latter
    feature permits TVB to be setup as a multi-user application. In this
    configuration, a login system enables users to access their personal
    sessions; by default their projects and data are private, but they can be
    shared with other users.} The graphical user interface (GUI) is web based,
    making use of HTML 5, WebGL, CSS3 and Java  Script \citep{Bostock_2011}
    tools to provide an intuitive and responsive  interface that can be
    locally and remotely accessed.
    
    \subsubsection{Web-based GUI}
    TVB provides a web-based interactive framework to generate, manipulate and
    visualize connectivity and network dynamics. Additionally, TVB comprises a
    set of classic time-series analysis tools, structural and functional
    connectivity analysis tools, as well as parameter exploration facilities
    which can launch \change{parallel simulations}{simulations in parallel} on
    a cluster or on multiple compute  cores of a server. The GUI of TVB has 6
    main working areas: \textbf{USER},  \textbf{PROJECT}, \textbf{SIMULATOR},
    \textbf{ANALYZE}, \textbf{STIMULUS}  and \textbf{CONNECTIVITY}. In
    \textbf{USER}, the users manage their  accounts and TVB settings. In
    \textbf{PROJECT}, individual projects are  managed and navigation tools
    are provided to explore their structure as  well as the data associated
    with them. \add{A sub-menu within this area provides a dashboard with a
    list of all the operations along with their current status (running, error,
    finished), owner, wall-time and associated data, among other information}.
    In \textbf{SIMULATOR} the  large-scale network model is set up and
    simulations launched, additional  viewers for structural and functional
    data are offered in 2D and 3D, as  well as other displays to visualize the
    results of a simulation. A history  of simulations is also available in
    this area. In \textbf{ANALYZE} time-series and
    network analysis methods are provided. In \textbf{STIMULUS}, users can
    interactively create stimulation patterns.  Finally, in
    \textbf{CONNECTIVITY}, users are given a responsive interface  to edit the
    connectivity matrices assisted by interactive visualization  tools. Fig.
    \ref{Fig02:tvb_wui} \change{summarizes the general usage of}{depicts the
    different working areas, as well as a number of their sub-menus, available
    through} the web UI. \add{A selection of screenshots illustrating the
    interface in a web browser is given in Fig.} \ref{Fig03:ui_screenshots}.

%%%--------------------------------FIGURE--------------------------------------%%
%% Fig. 2
  
%%--------------------------------FIGURE--------------------------------------%%
% Fig. 3

%##--------------------------------------------------------------------------##%
%##                           DATA                                           ##%
%##--------------------------------------------------------------------------##% 
    
    \subsubsection{Data Management and Exchange}
    
    One of the goals of TVB is to allow researchers from different backgrounds 
    and with different programming skills to have quick access to their 
    simulated data. Data from TVB can be exchanged with other instances of TVB
    (copies installed on different computers) or with other applications in the 
    neuroscientific community, e.g. MATLAB, Octave, The Connectome ToolKit 
    \citep{Gerhard_2011}. 
    
    \paragraph*{Export} A project created within TVB can be entirely exported 
    to a .zip file. Besides storing all the data generated within a particular
    project in binary files, additional XML files are created to provide a 
    structured storage of metadata, especially with regard to the steps taken 
    to set up a simulation, configuration parameters for specific operations,
    time-stamps and user account information. This mechanism procures a summary
    of the procedures carried on by researchers within a project which is used
    for sharing data across instances of TVB. The second export mechanism 
    allows the export of individual data objects. The data format used in TVB 
    is based on the HDF5 format \remove{(The HDF Group. Hierarchical data format 
    version 5)} \citep{HDF5} because it presents a number of advantages over 
    other formats: 
        1) huge pieces of data can be stored in a condensed form; 
        2) it allows grouping of data in a tree structure;
        3) it allows metadata assignment at every level; and 
        4) it is a widely used format, accessible in several programming 
        languages and applications.
    Additionally, each object in TVB has a global unique identifier (GUID) 
    which makes it easy to identify an object across systems, avoiding naming 
    conflicts among files containing objects of the same type.
    
    \paragraph*{Import}
    
    A set of mechanisms ("uploaders") is provided in TVB to import data into the
    framework, including neuroimaging data generated independently by other
    applications. The following formats are supported: NIFTI-1 (volumetric time-
    series), GIFTI (surfaces) and CFF (connectome file). General compression formats,
    such as ZIP and BZIP2 are also supported for certain data import routines that
    expect a set of ASCII text files compressed in an archive. Hence the use of
    general compression formats means that preparing datasets for TVB is as simple
    as generating an archive with the correct ASCII files, in contrast to some of
    the other neuroscientific data formats found elsewhere. For instance, a
    \tvbdatatype{Connectivity} dataset (\emph{connectome}) may be uploaded as a zip
    folder containing the following collection of files: (i) areas.txt, (ii)
    average\_orientations.txt, (iii) info.txt, (iv) positions.txt, (v)
    tract\_lengths.txt and (vi) weigths.txt. More conventions     and guidelines to
    use each uploader routine can be found in the \emph{User Guide} of TVB's
    documentation.

    \subsubsection{File Storage}
    The storage system is a tree of folders and files. The actual location on
    disk is configurable by the user, but the default is a folder called
    ``TVB'' in the user's home folder. There is a sub-folder for each
    \textbf{Project}  in which an XML file containing details about the
    project itself is stored.  Then for each operation, one folder per
    operation is created containing a  set of .h5 files generated during that
    particular operation, and one XML  file describing the operation itself.
    The XML contains tags like   \emph{creation date}, \emph{operation status}
    (e.g. Finished, Error),  \emph{algorithm reference}, \emph{operation
    GUID}, and most importantly \emph{input parameters dictionary}.
    Sufficiently detailed information is stored in the file system to be able
    to export data from one instance of TVB and to then import it into another
    instance,  correctly recreating projects, including all operations and
    their results. 
    Even though the amount of data generated per operation varies greatly,
    since it depends strongly on the Monitors used and parameters of the 
    simulation, some rough estimates are given below:

    - A 1000 ms long, region-based simulation with all the default parameters
    requires approximatively 1 MB of disk space.

    - A 10 ms long, surface-based simulation, using a precalculated sparse
    matrix to describe the local connectivity kernel and all the default
    parameters, requires about 280 MB.

    Users can manually remove unused data using the corresponding controls in
    TVB's GUI. In this case, all files related to these data are also deleted,
    freeing disk space. The amount of physical storage space available to TVB
    can be configured in the \textbf{USER -> Settings} working area of the GUI
    -- this is, of course, limited by the amount of free space available on the
    users hard drives.

    \subsubsection{Database Management System}
      % Why SQLite and PostgreSQL?
    Internally, TVB framework uses a relational database (DB), for ordering and 
    linking entities and as an indexing facility to quickly look up data. 
    At install time, users can choose between SQLite 
    (a file based database and one of the most used embedded DB systems) and 
    PostgreSQL (a powerful, widely spread, open-source object-relational DB 
    system which requires a separate installation by users) as the DB engine. 
    In the database, only references to the entities are stored, with the actual 
    operation results always being stored in files, due to size. A relational 
    database was chosen as it provides speed when filtering entities and 
    navigating entity relationship trees.
    
    \subsection{TVB Datatypes}\label{subsection:tvb-datatypes}    
    In the architecture of TVB, a middleware layer represented by
    TVB-\tvbdatatype{Datatypes} allows the handling and flow of data between
    the scientific kernel and the supporting framework.
    TVB-\tvbdatatype{Datatypes}  are annotated data structures which contain
    one or more data attributes and  associated descriptive information, as
    well as methods for operating on the  data they contain. The definition of
    a \tvbdatatype{Datatype} is achieved using  TVB's traiting system, which
    was inspired by the traiting system developed by  Enthought
    \citep{EnthoughtTraits}. The traiting system of TVB, among other things,
    provides a mechanism for annotating data, that is, associating additional
    information with the data which is itself usually a single number or an
    array of  numbers. A complete description of TVB's traiting system is
    beyond the  scope of this article. However, in describing TVB's
    \tvbdatatype{Datatypes} we will give an example of its use, which should
    help to provide a basic understanding of the mechanism.
      
    A number of basic TVB-\tvbdatatype{Datatypes} are defined based on
    \tvbtrait{Types} that are part of the traiting system, with these traited
    \tvbtrait{Types}, in turn, wrapping Numpy data types. For instance,
    TVB-\tvbdatatype{FloatArray} is a datatype derived from the traiting
    system's  \tvbtrait{Array} type, which in turn wraps Numpy's
    \textbf{ndarray}.  The traiting system's \tvbtrait{Array} type has
    attributes or annotations,  such as: \tvbattribute{dtype}, the numerical
    type of the data contained in the array; \tvbattribute{label}, a short
    (typically one or two word)  description of what the \tvbtrait{Array}
    refers to, this information is  used by the supporting framework to create
    a proper label for the GUI;  \tvbattribute{doc}, a longer description of
    what the \tvbtrait{Array}  refers to, allowing the direct integration of
    useful documentation into array objects; and \tvbattribute{default}, the
    default value for an  instance of an \tvbtrait{Array} type. In the case of
    a  \tvbdatatype{FloatArray}, the \tvbattribute{dtype} attribute is fixed
    as  being numpy.float64.
    
    More complex, higher-level, TVB-\tvbdatatype{Datatypes} are then built up
    with  attributes that are themselves basic TVB-\tvbdatatype{Datatypes}.
    For example, TVB-\tvbdatatype{Connectivity} is datatype which includes
    multiple  \tvbdatatype{FloatArray}s, as well as a number of other traited
    types,  such as \tvbtrait{Integer} and \tvbtrait{Boolean}, in its
    definition.  An example of a \tvbdatatype{FloatArray} being used to define
    an attribute  of a \tvbdatatype{Connectivity} can be seen in Code
    \ref{code:floatarray}.  The high-level \tvbdatatype{Datatypes} currently
    defined in TVB are summarized in Table 2. %\ref{tab:tvb_datatypes}.
    

%%--------------------------------CODE--------------------------------------%%
\pagebreak
\begin{lstlisting}[backgroundcolor=\color{black!5}, 
                   caption= An instance of TVB's \tvbdatatype{FloatArray} 
                    \tvbdatatype{Datatype} being used to define the conduction
                    \tvbattribute{speed} between brain regions as an attribute
                    of a \tvbdatatype{Connectivity} \tvbdatatype{Datatype} \\,
                   commentstyle=\itshape\color{green!50!black},
                   frame=single,
                   stringstyle=\color{magenta},
                   keywordstyle={\bf\ttfamily\color{blue}},
                   label=code:floatarray,
                   literate=%
                    {0}{{{\color{orange}0}}}1
                    {3.}{{{\color{orange}3.}}}1,
                   morekeywords={*,FloatArray},
                   showspaces=false,
                   showtabs=false]
                   
speed = FloatArray(
    label = "Conduction speed", 
    default = numpy.array([3.0]),
    doc = """A single number or matrix of conduction speeds for the 
             myelinated fibre tracts between regions.""")

\end{lstlisting} 
    
    
    \change{As an example of the flow of data through TVB using
    \tvbdatatype{Datatypes}. } {An example indicating the usage and features
    of TVB-\tvbdatatype{Datatypes} is provided below. } When a user uploads a
    connectivity dataset through the UI, an instance of  a
    \tvbdatatype{Connectivity} datatype is generated. This
    \tvbdatatype{Connectivity} datatype is one of the required input arguments
    when creating an instance of \change{the simulation
    engine}{\tvbmodule{Simulator}}. As a  result of the execution of a
    simulation, other TVB-\tvbdatatype{Datatypes} are generated, for instance
    one or more \tvbdatatype{TimeSeries} datatypes. Specifically, if the
    simulation is run using  the MEG and EEG recording modalities then
    \tvbclass{TimeSeriesMEG},  \tvbclass{TimeSeriesEEG}, which are subclasses
    of  \tvbdatatype{TimeSeries}, are returned. Both the
    \tvbdatatype{Connectivity} and \tvbdatatype{TimeSeries} datatypes are
    accepted by a range of appropriate analysis and visualization methods.
    
    Further, TVB-\tvbdatatype{Datatypes} have attributes and metadata which
    remains accessible after exporting in TVB format. The metadata includes a 
    technical description of the data (storage size for instance) as well as
    scientifically relevant properties and useful documentation to properly 
    interpret the dataset. In the shell interface, the attributes of 
    TVB-\tvbdatatype{Datatype} can be accessed by their key-names in the same
    way as Python dictionaries. 

%%--------------------------------FIGURE--------------------------------------%%
%% Fig. 4
    

\begin{center}     \begin{table*}[ht]\tiny     \label{tab:tvb_datatypes}
\caption{TVB Datatypes. Specifications about the requirements to build a
TVB-Datatype can be found in the documentation of the distribution packages.}
\begin{tabularx}{\textwidth}{lll}             \toprule             Base Class
Datatype & Description                   & Derived Classes \\
\midrule             Connectivity        & Maps connectivity matrix data     &
Connectivity\\             \\             Surfaces            & Covers surface
representations    &   CorticalSurface,
SkinAir,
BrainSkull,
SkullSkin,
EEGCap,
FaceSurface,
Cortex,
RegionMapping, \\
&&  LocalConnectivity\\             \\
Volumes             & Wraps volumetric data            & ParcellationMask,
StructuralMRI \\             \\
Sensors              & Wraps sensors data used in different acquisition
techniques to generate physiological recordings
& SensorsEEG, SensorsMEG, SensorsInternal\\             \\
ProjectionMatrix    & Wraps matrices defining a linear operator to map the
spatial sources into the leadfield domain.
&  ProjectionRegionEEG, ProjectionSurfaceEEG, ProjectionRegionMEG   \\
            
                              & It relates two datatypes: a source of type
Connectivity or Surface and a set of Sensors.
& ProjectionSurfaceMEG \\                               & The matrix is
computed using OpenMEEG. \citep{Gramfort_2010} & \\             \\
Equations           & Commonly used functions for defining local connectivity
kernels and stimulation patterns. &    \\             \\
SpatialPattern      & Contains patterns mainly used as stimuli. It makes use
of Equation datatypes
& SpatioTemporalPattern,
StimuliRegion,
StimuliSurface,
SpatialPatternVolume \\             \\
TimeSeries          & One of the most important TVB-Datatypes. Derived classes
wrap measurements recorded  & TimeSeriesRegion,
TimeSeriesSurface,
TimeSeriesVolume,
TimeSeriesEEG, \\                              & under different acquisition
modalities.                           &   TimeSeriesMEG \\             \\
Graph               & Wraps results from a covariance analysis or results from
BCT analyzers    &   Covariance,
ConnectivityMeasure\\             \\
MappedValues      & Wraps a single value computed from a TimeSeries object &
\\             ModeDecomposition & Wraps results from matrix factorization
analysis (i.e. PCA and ICA) &   PrincipalComponents,
IndependentComponents\\             \\
Spectral            & Wraps results from frequency analysis         &
FourierSpectrum, WaveletCoefficients, ComplexCoherenceSpectrum \\
\bottomrule         \end{tabularx}     \end{table*} \end{center}


%##--------------------------------------------------------------------------##%
%##                           SIMULATOR                                      ##%
%##--------------------------------------------------------------------------##% 

    \subsection{TVB Simulator}
    
    The simulation core of TVB brings together a mesoscopic model of neural
    dynamics with structural data. \add{The latter defines both the spatial
    support} (see Fig. \ref{Fig04:connectivity})\add{, upon which the brain
    network model is built, and the hierarchy of anatomical connectivity, that
    determines the spatial scale represented by the structural linkages
    between nodes} \citep{Freeman_1975book}. \change{It}{Simulations} then
    recreate the emergent brain  dynamics by numerically integrating this
    coupled system of differential equations. All these entities have their
    equivalent representation as  \tvbclass{classes} either in the scientific
    \tvbmodule{modules} or  \tvbdatatype{datatypes}, and are bound together in
    an instance of the  \tvbclass{Simulator} class. \add{In the following
    paragraphs we describe all the individual components required to build a
    minimal representation of a brain network model and run a simulation, as 
    well as the outline of the operations required to
    initialize a \tvbclass{Simulator} object and the operations of the update
    scheme.} \remove{This class is defined in the  \tvbmodule{simulator}
    module and has several \tvbmethod{methods} to set up the spatiotemporal
    dimensions of the input and output arrays, based on configurable
    attributes of the individual components such as integration time step
    (e.g.
    \tvbmodule{integrators}.\tvbclass{HeunDeterministic}.\tvbattribute{dt}),
    structural spatial support (e.g.
    \tvbdatatype{connectivity}.\tvbclass{Connectivity} or
    \tvbdatatype{surfaces}.\tvbclass{CorticalSurface}) and transmission speed
    (e.g.
    \tvbdatatype{connectivity}.\tvbclass{Connectivity}.\tvbattribute{speed})
    as well as a cascade of specific \tvbmethod{configuration methods} to
    initialize all the individual components required to build a minimal
    representation of a brain network model and perform a
    simulation.}

%%--------------------------------FIGURE--------------------------------------%%
% Fig. 5    
    
    \subsubsection{Coupling}
    The brain activity (state variables) that has been propagated over the 
    long-range \tvbdatatype{Connectivity} pass through these functions before
    entering the equations of a \tvbclass{Model} describing the local dynamics.
    A \tvbclass{Coupling} function's primary purpose is to rescale the
    incoming activity to a level appropriate to the population model. The base
    \tvbclass{Coupling} class as well as a number of different coupling
    functions are implemented in the \tvbmodule{coupling} module, for instance
    \tvbclass{Linear} and \tvbclass{Sigmoidal}. 
    
    \subsubsection{Population Models}
    % The thing class thing model metaphor
    A set of default mesoscopic neural models are defined in TVB's 
    \tvbmodule{models}. All these models of local dynamics are classes derived
    from a base \tvbclass{Model} class. 
    
    We briefly discuss the implemented population models in order of
    increasing complexity. They include a generic two dimensional oscillator,
    a collection of classical  population models and two recently developed
    multi-modal neural mass models. Below, $N$ refers to the number of state
    variables or equations governing the  evolution of the model's temporal
    dynamics; $M$ is the number of modes and by default $M=1$ except for the
    multi-modal models.
    
    The \tvbclass{Generic2dOscillator} model ($N=2$) is a generic phase-plane
    oscillator model capable of generating a wide range of phenomena observed
    in neuronal population dynamics, such as multistability, the coexistence
    of oscillatory and non-oscillatory dynamics, as well as displaying
    dynamics at multiple  time scales.
    
    The \tvbclass{WilsonCowan} model \citep{Wilson_1972} ($N=2$) describes the 
    firing rate of a neural population consisting of two subpopulations (one 
    excitatory and the other inhibitory). It was originally derived using 
    phenomenological arguments. This neural mass model provides an intermediate 
    between a microscopic and macroscopic level of description of neural 
    assemblies and populations of neurons since it can be derived from 
    pulse-coupled neurons \citep{Haken_2001} and its continuum limit 
    resembles neural field equations \citep{Jirsa_1996}.  
    
    The \tvbclass{WongWang} model \citep{Wong_2006} represents a reduced
    system of $N=2$ coupled  nonlinear equations, originally derived for
    decision making in two-choice tasks.  The \tvbclass{BrunelWang} model
    \citep{Brunel_2001, Brunel_2003} is a mean  field model derived from
    integrate-and-fire  spiking neurons and makes the approximation of
    randomly distributed interspike  intervals. It is notable that this
    population model shows only attractor states of  firing rates. It has been
    extensively used to  study working memory. Its complexity resides in the
    number of parameters  that it uses to characterize each population
    ($N=2$). These parameters  correspond to physical quantities that can be
    measured in neurophysiology  experiments. The current implementation of
    this model is based on the approach  used in \citep{Deco_2012}.
    
    
    The \tvbclass{JansenRit} model \citep{Jansen_1995} is a derivative of the
    Wilson-Cowan model and features three coupled subpopulations of cortical
    neurons: an excitatory population of pyramidal cells interacting with two
    populations of interneurons, one inhibitory and the excitatory.  This
    model can produce alpha activity consistent with that measured in EEG, and is
    capable of simulating evoked potentials \citep{Jansen_1993}. It displays a
    surprisingly rich and complex oscillatory dynamics under periodic
    stimulation  \citep{Spiegler_2010}. Each population is described by a
    second order differential equation. As a consequence the system is
    described by a set of $N=6$ first order differential equations.
    
    
    The \tvbclass{StefanescuJirsa2D} and \tvbclass{StefanescuJirsa3D} models
    \citep{Stefanescu_2008, Jirsa_2010, Stefanescu_2011} are neural mass
    models derived from a globally coupled population of neurons of a
    particular kind. The  first one has been derived from coupled 
    FitzHugh-Nagumo neurons  \citep{FitzHugh_1961, Nagumo_1962}, which, with $N=2$, are
    capable of  displaying excitable dynamics, as well as oscillations. The
    second is  derived from coupled Hindmarsh-Rose neurons
    \citep{Hindmarsh_1984}, which are also capable of producing excitable and
    oscillatory dynamics, but  with $N=3$ have the additional capability of
    displaying transient  oscillations and bursts. The two Stefanescu-Jirsa
    models show the most complex  repertoire of dynamics (including bursting
    and multi-frequency oscillations).  They have been derived using mean
    field techniques for parameter dispersion  \citep{Assisi_2005} and have an
    additional dimension, the mode $M$, which partitions  the dynamics into
    various subtypes of population behavior. These models are therefore
    composed of 12 ($N=4$, $M=3$) and 18 ($N=6$, $M=3$) state variables,
    respectively.
    
    \subsubsection{Integrators}
    The base class for integration schemes is called \tvbclass{Integrator}, an
    \tvbmodule{integrators} module contains this base class along with a set
    of  specific integration scheme classes for solving both deterministic and
    stochastic differential equations. The specific schemes implemented for
    brain network simulations include the \tvbclass{Euler} and \tvbclass{Heun}
    methods.\remove{The 4th-order \tvbclass{Runge-Kutta} method is available,
    however, this is only suitable for the integration of single (isolated)
    instances of the dynamic \tvbclass{Models}.}\add{ The 4th-order 
    \tvbclass{Runge-Kutta} (rk4) method is only available for solving ordinary differential
    equations (ODEs), i.e., deterministic integration, given that there are
    various variants for the stochastic version of the method, differing rates of
    convergence being one of the points that several attempts of creating a
    stochastic adaptation fail at (see} \citet{Burrage_2004} for an overview). 
    Therefore, this method is available for drawing example trajectories in the interactive
    phase-plane plot tool. \remove{(Fig. 8).} 
    
    
    \subsubsection{Noise}
    
    Noise plays a crucial role for brain dynamics, and hence for brain
    function \citep{McIntosh_2010}. The \tvbmodule{Noise} module consists of
    two base classes: \tvbclass{RandomStream} that wraps Numpy's RandomState
    class and \tvbclass{Noise}. The former provides the ability to create
    multiple random streams which can be independently seeded or set to
    an explicit initial state.  The latter is the base class from which
    specific noises, such as white and colored \citep{Fox_1988},  are
    derived. In TVB's implementation \tvbclass{Noise} enters as an additional
    term within the stochastic integration schemes, and can be either an
    \tvbclass{Additive} or \tvbclass{Multiplicative} process
    \citep{Kloeden_1995}.  As well as providing a means to generate
    reproducible stochastic processes for the  integration schemes, the
    related classes in \tvbmodule{noise} are  used to set the initial
    conditions of the system when no explicit initial  conditions \add{are
    specified}.
    
    \subsubsection{Monitors}
    The data from a simulation is processed and recorded while the simulation
    is running, that is, while the differential equations governing the system
    are being integrated. The base class for these processing and recording
    methods is the \tvbclass{Monitor} class in the \tvbmodule{Monitors}
    module.  We consider two main types of  online-processing: 1) raw or 
    low-level; and 2) biophysical or high-level. The output of a
    \tvbclass{Monitor} is a 4-dimensional  array (which can be wrapped in the
    corresponding \tvbdatatype{TimeSeries}  datatype), i.e., a 3D state vector
    as a function of time. For the first  kind of \tvbclass{Monitors} these
    dimensions correspond to  $[time, state\,variables, space, modes]$ where
    ``space'' can be either  brain regions or vertices of a cortical surface
    plus non-cortical brain  regions. The number of state variables as well as
    the number of modes strictly depend on the \tvbclass{Model}. For the
    second kind of  \tvbclass{Monitors}, the dimensions are $[time, 1,
    sensors, 1]$. The  simplest form of low-level \tvbclass{Monitor} returns
    all the simulated  data, i.e., time points are returned at the sampling
    rate corresponding to the integration scheme's step size and all 
    state variables are returned for all nodes. All other low-level
    \tvbclass{Monitors} perform some degree of down-sampling, such as
    returning only a reduced set state variables (by  default the
    \emph{variables of interest} of a \tvbclass{Model}), or  down-sampling in
    ``space'' or time. Some variations include temporally sub-sampled,
    spatially averaged and temporally sub-sampled, or temporally averaged. The
    biophysical \tvbclass{Monitors} instantiate a physically realistic
    measurement process on the simulation, such as \tvbclass{EEG},
    \tvbclass{MEG}, \add{\tvbclass{SEEG}} or \tvbclass{BOLD}. For the first
    two, a \tvbdatatype{ProjectionMatrix} is also required. This matrix  maps
    source activity (``space")  to sensor activity (``sensors").
    \remove{Currently,} OpenMEEG \citep{Gramfort_2010} \change{is}{was} used
    to generate the demonstration projection matrix, also known as 
    lead-field or gain matrix, that correspond to the EEG/MEEG forward solution.
    \add{The forward solution modeling the signals from depth electrodes is
    based on the point dipole model in homogeneous space} \citep{Sarvas_1987}.
    The \tvbclass{BOLD} monitor is based on \citet{Buxton_1997} and \citet{Friston_2000}.
    Fig. \ref{Fig05:tvb_sim} summarizes the fundamental blocks  required to
    configure a full model, launch a simulation and retrieve the simulated
    data. 

    In most neural mass models there is a state variable representing some type
    of neural activity (firing rate, average membrane potential, etc.), which
    serves as a basis for the biophysical monitors. The state variables used as
    source of neural activity depend both on the \tvbclass{Model} and the
    biophysical space that it will be projected onto (MEG, EEG, BOLD). Given a
    neural mass model with a set of state variables, G-Users can choose which
    subset of state variables will be fed into a \tvbclass{Monitor}
    (independently for each monitor). However, how a given \tvbclass{Monitor}
    operates on this subset of state variables is an intrinsic property of the
    monitor. Users with programming experience can, of course, define new
    monitors according to their needs. Currently, there is not a mechanism
    providing automatic support for general operations over state variables
    before they are passed to a monitor. As such, when the neural activity
    entering into the monitors is anything other than a summation or average
    over state variables then it is advised to redefine the \tvbclass{Model} in
    a way that one of the state variables actually describes the neural activity
    of interest.

	\note[PSL]{New subsection : workflow of the simulator algorithm}
    \subsubsection{Outline of the simulation algorithm}
    
    The \tvbclass{Simulator} class has several
    \tvbmethod{methods} to set up the spatiotemporal dimensions of the input
    and output arrays, based on configurable attributes of the individual
    components such as integration time step (e.g.
    \tvbmodule{integrators}.\tvbclass{HeunDeterministic}.\tvbattribute{dt}),
    structural spatial support (e.g.
    \tvbdatatype{connectivity}.\tvbclass{Connectivity} or
    \tvbdatatype{surfaces}.\tvbclass{CorticalSurface}) and transmission speed
    (e.g.
    \tvbdatatype{connectivity}.\tvbclass{Connectivity}.\tvbattribute{speed})
    as well as a cascade of specific \tvbmethod{configuration methods} to
    interface them. The \tvbclass{Simulator} class coordinates the
    collection of objects from all the modules in the scientific library
    needed to build the network model and yield the simulated data.  To perform a
    simulation a \tvbclass{Simulator} object needs to be: 1)  configured,
    initializing all the individual components and calculating attributes
    based on the combination of objects passed to the \tvbclass{Simulator} 
    instance; and 2) called in a loop to obtain simulated data, i.e., to run 
    the simulation (see Code \ref{code:region_deterministic}). The next 
    paragraphs list the main operations of the simulation algorithm. 

    \paragraph*{Initializing a}\tvbclass{Simulator}
      \begin{enumerate}

        \item Check if the transmission \tvbattribute{speed} was provided. 

        \item Configure the \tvbdatatype{Connectivity} matrix (connectome). The
        \tvbattribute{delays} matrix is computed using the distance matrix and
        the transmission \tvbattribute{speed}. Get the \tvbattribute{number of
        regions}.

        \item Check if a \tvbdatatype{Surface} is provided.

        \item Check if a stimulus \tvbdatatype{pattern} is provided.

        \item Configure individual components: \tvbclass{Model},
            \tvbclass{Integrator}, \tvbclass{Monitors}. From here we obtain
            \tvbattribute{integration time step size}, 
            \tvbattribute{number of statevariables}, \tvbattribute{number of modes}.

        \item Set the \tvbattribute{number of nodes} 
            (region-based or surface-based 
            simulation). If a \tvbdatatype{Surface} was given the 
            \tvbattribute{number of nodes} will correspond to the number of 
            vertices plus the number of non-cortical regions, otherwise it will 
            be equal to the \tvbattribute{number of regions} in the 
            \tvbdatatype{Connectivity} matrix.

        \item Spatialise model parameters if required. Internally, TVB uses 
            arrays for model parameters, if the size of the array for a 
            particular parameter is 1, then the same numerical value is applied
            to all nodes. If the size of the parameter array is $N$, where $N$ is 
            the number of nodes, the parameter value for each node is taken 
            from the corresponding element of the array of parameter values.

        \item If applicable, configure spatial component of stimulation 
            \tvbdatatype{Patterns} (requires \tvbattribute{number of nodes}).

        \item Compute \tvbattribute{delays} matrix in integration time steps.

        \item Compute the \tvbattribute{horizon} of the delayed state, that is 
            the maximum delay in integration time steps.

        \item Set the \tvbattribute{history} shape. The history state contains 
            the activity that propagates from the delayed state to the next. 

        \item Determine if the \tvbclass{Integrator} is deterministic or 
            stochastic. If the latter, then configure the \tvbclass{Noise} and 
            the integration method accordingly. 

        \item Set \tvbattribute{initial conditions}. This is the \tvbattribute{
              state} from which the simulation will begin. If none is provided,
              then random initial conditions are set based on the ranges of the
              model's state variables. Random initial conditions are fed to the
              initial \tvbattribute{history} array providing the minimal state of
              the network with time-delays before $t=0$. If initial conditions are
              user-defined but the length along the time dimension is shorter than
              the required \tvbattribute{horizon}, then  the
              \tvbattribute{history} array will be padded using the same method of
              described for random \tvbattribute{initial conditions}.

        \item Configure the monitors for the simulation. Get 
            \tvbattribute{variables of interest}.
      \end{enumerate}
    
    \paragraph*{Calling a }\tvbclass{Simulator} 
      \begin{enumerate}

        \item Get \tvbattribute{simulation length}.

        \item Compute estimates of run-time, memory usage and storage.

        \item Check if a particular \tvbattribute{random state} was provided 
            (random seed). This feature is useful for reproducibility of results, 
            for instance, getting the same stream of random numbers for the 
            \tvbclass{Noise}.

        \item Compute the \tvbattribute{number of integration steps}. 

        \item If the simulation is surface-based, then get attributes required
            to compute \tvbdatatype{Local Connectivity} kernel.

        \item Update state loop

          \begin{enumerate}
            \item Get the corresponding coupled delayed activity. That is, 
                compute the the dot product between the weights matrix 
                (connectome) and the delayed state of the 
                \tvbattribute{coupling variables}, transformed by a 
                (long-range) \tvbclass{Coupling} function.
          
            \item Update the \tvbattribute{state} array. This is the numerical 
                integration, i.e., advancing an integration time step, of the 
                differential equations defining the neuron model. Distal 
                delayed activity, local instantaneous activity and stimulation 
                are fed to the integration scheme. 
    
            \item Update the \tvbattribute{history}

            \item Push state data onto the \tvbclass{Monitors}. Yield any processed 
                time-series data point if available. 
          \end{enumerate}
      \end{enumerate}

    
    
    
    As a working example, in Code \ref{code:region_deterministic}, we show a 
    code snippet which uses TVB's scripting interface and some of the classes
    and modules we have just described to generate one second of brain activity.
    \add{The for loop in the example code allows scripting users to receive 
    time-series data as available and separately for each of the monitors processing
simulated raw data. In this implementation, at each time step or certain
number of steps, data can be directly stored to disk, reducing the memory
footprint of the simulation. Such a feature is particularly useful when
dealing with larger simulations. Likewise, data can be accessed while the
simulation is still running, which proves to be advantageous for modeling
paradigms where one of the output signals is fed back to the network model as
stimulation for instance (see the paragraph about \emph{Dynamic modeling} in
section} \ref{sec:PerformanceReproducibilityFlexibility}).
     
    
%##-------------------------------------------------------------------##
%#                             CODE                                   ##
%##-------------------------------------------------------------------##    
    % uncoupled a = 0.000042
    % coupled   a = 0.0042
%\pagebreak
\begin{lstlisting}[backgroundcolor=\color{black!5}, 
                   caption=Script example to simulate 1 second of brain 
                   activity. Output is recorded with two different monitors.\\,
                   commentstyle=\itshape\color{green!50!black},
                   frame=single,
                   keywordstyle={\bf\ttfamily\color{blue}},
                   label=code:region_deterministic,
                   showspaces=false,
                   showtabs=false]
    
from tvb.simulator.lab import *
        
#Initialise a Model, Connectivity and Global Coupling
oscilator = models.Generic2dOscillator()
white_matter = connectivity.Connectivity()
white_matter.speed = numpy.array([4.0]) # [mm/ms]
white_matter_coupling = coupling.Linear(a=0.0042)

#Initialise an Integrator
heunint = integrators.HeunDeterministic(dt=2**-4) 

#Initialise some Monitors with period in physical time
mon_raw = monitors.Raw()
mon_tav = monitors.TemporalAverage(period=2**-2)
what_to_watch = (mon_raw, mon_tav)

#Initialise a Simulator object
sim = simulator.Simulator(model = oscilator, 
                          connectivity = white_matter,
                          coupling = white_matter_coupling, 
                          integrator = heunint, 
                          monitors = what_to_watch)

# Configure the Simulator object
sim.configure()
LOG.info("Starting simulation...")
    
raw_data,  raw_time  = [], []
tavg_data, tavg_time = [], []
    
# Call the Simulator object -- Run simulation
for raw, tavg in sim(simulation_length=2**10):
    if not raw is None:
        raw_time.append(raw[0])
        raw_data.append(raw[1])
    if not tavg is None:
        tavg_time.append(tavg[0])
        tavg_data.append(tavg[1])
LOG.info("Finished simulation.")
\end{lstlisting}
 
    \subsection{Analyzers and Visualizers}
    
    For the analysis and visualisation of simulated neuronal dynamics as well
    as imported data, such as anatomical structure and experimentally
    recorded time-series, several algorithms and techniques are currently
    available in  TVB. Here we list some of the algorithms and methods that
    are provided to perform analysis and visualization of data through the GUI.
    
    \paragraph{Analyzers \textcolon} are mostly standard algorithms for 
    time-series and network analysis. The analyzers comprise techniques 
    wrapping functions from Numpy (Fast Fourier Transform (FFT), 
    auto-correlation, variance metrics), Scipy (cross-correlation), 
    scikit-learn (ICA) \citep{scikit-learn} and matplotlib-mlab (PCA) 
    \citep{Hunter_2007}. In addition, there are specific implementations of
    the wavelet transform, complex coherence \citep{Nolte_2004a,Freyer_2012a} 
    and multiscale entropy (MSE) \citep{Costa_2002a, Costa_2005, Lake_2011}.
    
    \paragraph{Visualizers \textcolon} are tools designed to correctly handle 
    specific \tvbdatatype{datatypes} and display their content. Representations currently
    available in the GUI include: histogram plots (Fig. 6A); 
    interactive time-series plots, EEG (Fig. 6C); 2D head topographic
    maps (Fig. 6B);  3D displays of surfaces and animations 
    (Fig. 6D) and network plots. Additionally, for shell users
    there is a collection of plotting tools available  based on matplotlib and
    mayavi \citep{mayavi_2011}.
    
%%--------------------------------FIGURE--------------------------------------%%
%% Fig. 6



\end{methods}   
%##--------------------------------------------------------------------------##%
%##                Performance, Reproducibility and Flexibility              ##%
%##--------------------------------------------------------------------------##%
    
    \section{Performance, Reproducibility and Flexibility}\label{sec:PerformanceReproducibilityFlexibility}
    \paragraph*{Testing for speed}\remove{We ran simulations for all possible
    combinations of two parameters:     simulation length and integration time step.
    These parameters, along with     node count and dimensionality of the
    \tvbclass{Model} dynamics, are the     ones that impose the most crucial
    constraints regarding memory usage and execution time.} In the context of
    full brain models there is no other platform against which we could compare
    the performance results for TVB and define a good ratio run-time/real-time.
    \add{As a first approximation a simple network of 74 nodes, whose node dynamics
    were governed by the equations of the \tvbclass{Generic2dOscillator} model
    (see Code }\ref{code:model_generic}) \add{was implemented in the Brian
    spiking neural network simulator. The integration step size was 0.125
    milliseconds ($dt=2^{-3}$ ms) and the simulation length was 2048 milliseconds. This network was
    evaluated without time delays and using a random sparse connectivity matrix.
    Execution times were about 4.5 seconds in Brian and 15 seconds in TVB. In
    contrast, when heterogeneous time delays were included, running times of
    the simulations implemented in Brian increased considerably (approximately 6.5x)
    whereas in TVB they hardly changed (approximately 1.2x). Simulations were
    run on a CPU Intel\textregistered Xeon \textregistered W3520 @ 2.67GHz. These
    results, although informative, expose the fact that the architectures of TVB and
    the Brian simulator are different and therefore they have been optimized
    accordingly to serve distinct purposes from a modeling point of view. }     
    % 6 MB @ Mac     
    % 8 MB L3 Cache @ Workstation     
    % 12 MB L3 Cache @ server nodes

    \add{To assess the performance of TVB in terms of simulation timings}, we
    also ran simulations for all possible combinations of two parameters:
    simulation length and integration time step (Fig. \ref{Fig07:ets}A). \remove{These parameters,
    along with node count and dimensionality of the \tvbclass{Model} dynamics,
    are the ones that impose the most crucial constraints regarding execution
    times and memory usage.} We made the following estimates: it takes on
    average 16 seconds to compute one second of brain network dynamics (at the
    region level\add{,} with an integration time step of 0.0625
    milliseconds ($dt=2^{-4}$ ms) \add{and including time delays of the order of 20 milliseconds which amounts to store about 320
    past states per time step)} on CPUs Intel \textregistered
    Xeon \textregistered X5672 @ 3.20GHz, CPU cache of 12MB and Linux kernel
    3.1.0-1-amd64 as operating system. \add{In Fig. }\ref{Fig07:ets}\add{B we quantify how running times
    increase as a function of the integration time step in 64 seconds long (region-based) 
    simulations for two different sizes of the connectivity matrix.}

    \add{In general, human cortical connectomes are derived from anatomical
    parcellations with a variable number of nodes, from less than 100 to over a
    few thousands nodes} \citep{Zalesky_2010a}\add{. Preliminary results of
    simulations (data not shown) using connectivity matrices of different sizes (16,
    32, 64, 128, 256, 512, 1024, 2048 and 4096 nodes) and a supplementary
    parameter (transmission speed that has an effect on the size of the
    \tvbattribute{history} array keeping the delayed states of the network)
    indicate that there is a quadratic grow of the running times for networks
    with more than 512 nodes. Since performance depends on a large number
    of parameters which have an effect on both memory (CPU cache and RAM) and CPU usage, and
    therefore resulting running times arise from the interaction between them,
    we see the need to develop more tests to stress in particular memory
    capacity and bandwidth in order to fully understand the aforementioned
    behavior.}


    \note[PSL]{new code with the equations of the generic model}

\begin{lstlisting}[backgroundcolor=\color{black!5}, 
                   caption=State equations of the generic plane 
                   oscillator as scripted to run the simulation 
                   in the Brian simulator. The description of the 
                   parameters are explained in the API documentation 
                   and will be discussed in the context of 
                   dynamical systems elsewhere.\\,
                   commentstyle=\itshape\color{green!50!black},
                   frame=single,
                   keywordstyle={\bf\ttfamily\color{blue}},
                   label=code:model_generic,
                   showspaces=false,
                   showtabs=false]
# model equations
eqs = '''
dV/dt = d * tau * (alpha * W - f * V**3 + e * V**2 + I)
dW/dt = d * (a + b * V + c * V**2 - beta * W) / tau
'''
\end{lstlisting}              

    In Future Work we talk about the approaches to benchmark and   improve the
    execution times of simulations. \remove{In future analysis, we intend to
    include additional dimensions, including the number of nodes in the network
    and a range of dynamic models.} For the present work we have restricted
    ourselves to present performance results looking at the parameters that have
    the strongest effect on simulations timings.
    
%%--------------------------------FIGURE--------------------------------------%%

%% Fig. 7 
  
    \note{Removed svn revision reference}
    
    \remove{\paragraph*{Reproducibility of results from the literature}}
    \add{\paragraph*{Reproducibility of results from the literature}}
    \citet{Ghosh_2008} and \citet{Deco_2009} demonstrated the important role
    of three large-scale parameters in the emergence of different cluster
    synchronization regimes: the global coupling strength factor, time-delays
    (introduced via the long-range connectivity fiber tract lengths and a
    unique transmission speed) and noise variance. \add{They built parameter
    space maps using the Kuramoto synchronization index}. Here, using TVB's
    scripting interface, we show it is easily possible to \change{replicate
    the scheme}{build a similar scheme} and perform a parameter space
    exploration in the coupling strength (\emph{gcs}) and transmission speed
    (\emph{s}) space. The \tvbdatatype{Connectivity} upon which the 
    large-scale network is built was the demonstration dataset. It is bi-hemispheric
    and  consists of 74 nodes, i.e., 37 regions per hemisphere. It includes
    all the cortical regions but without any sub-cortical structure such as
    the thalamic nuclei. Its weights are quantified by integer values in the
    range 0 to 3. The evolution of the local dynamics were represented by the
    model \emph{Generic2dOscillator}, configured in such a way that a single
    isolated node exhibited 40Hz oscillations (Fig.
    \ref{Fig08:results_reproduce_deco_pp}). The variance of the output 
    time-series was chosen as a simple, yet informative measure to represent the
    collective dynamics as a function of the parameters under study. Results
    are shown in Fig. \ref{Fig09:results}B. Parameter sweeps can also be
    launched from TVB web-interface (see Fig. \ref{Fig10:pse_ui} for an
    illustration). 

    \add{Currently TVB provides two scalar metrics based on the variance of
    the output time-series to perform data reduction when exploring a certain
    parameter space. These are \emph{Variance of the nodes Variances} and
    \emph{Global Variance}. The former zero-centers the output time-series and
    compute the variance over time of the concatenated time-series of each  state
    variable and mode for each node and subsequently the variance of the nodes
    variances is computed. This metric describes the variability of the temporal
    variance of each node. In the latter all the time-series are zero-centered and
    the variance is computed over all data points contained in the output array.}
    
    \add[PSL]{With this example we intended to expose the possibility to
    reproduce workflows, i.e. modeling schemes, found in the literature. TVB
    is a modeling platform providing a means of cross-validating scientific
    work by encouraging reproducibility of the results.}

%%--------------------------------FIGURE--------------------------------------%%
%% Fig. 8  
%%
%%    
%%--------------------------------FIGURE--------------------------------------%%
%% Fig. 9  
%%
%%
%%--------------------------------FIGURE--------------------------------------%%
%Fig. 10          
%%    
    \paragraph*{Higher-level simulation scenarios using stimulation protocols}
    As one possible use case, we have set up an example based on the scheme
    used in \citet{McIntosh_2010}. The goal is to demonstrate how to build
    stimulation patterns in TVB, use them in a simulation, obtain EEG 
    recordings of both the activity similar to the resting state \add{(RS)} 
    and to evoked responses (ER), and finally make a differential analysis 
    of the complexity of the resulting time-series by computing MSE.
    
    In vision neuroscience, the two-stream hypothesis \citep{Schneider_1969}
    suggests  the existence of two streams of information processing, the
    ventral and the dorsal  stream. In one of these pathways, the ventral
    stream, the activity from  subcortical regions project to V1 and the
    activity propagates to the temporal  cortices through V2 and V4
    \citep{Goodale_1992}. We systematically stimulated  the area corresponding
    to the primary visual cortex (V1) to demonstrate the  functioning of TVB
    stimulation \tvbdatatype{Patterns} and observed how the  the activity
    elicited by a periodic rectangular pulse propagates to neighboring
    regions, especially V2.
    
    \remove{To further demonstrate the many scenarios that can be 
    set up in TVB, we simulated the same network model without and with noise 
    (using deterministic Fig. 11A and stochastic Fig. 11B integration schemes. }  

%%--------------------------------FIGURE--------------------------------------%%
%% Fig. 11           
    
%--------------------------------FIGURE--------------------------------------%%
%% Fig. 12         
    
    Benefiting from TVB's flexibility we \change{have shown}{show} \add{in
    Fig.} \ref{Fig11:sampenAB} that it is possible to systematically
    stimulate \change{different brain areas (e.g. occipital region)}{a
    specific brain region (e.g. V1) and to highlight the anatomical connection
    to its target region (e.g. V2) by observing the arrival of the delayed
    activity}; analyze the responses of the model\remove{during both the
    resting and evoked states}; handle multi-modal simulated data; and extract
    metrics from computationally expensive algorithms \add{to characterize
    both the ``resting'' and ``evoked'' states}. \remove{Scripts to reproduce
    results Figs.11 and 12 are available in the distribution packages of TVB.}
    
    \add{Currently, TVB permits the stimulation and read-out of activity from
    any brain area defined in the anatomical parcellation used to derive the
    connectome. This modeling example was built imposing a strong restriction
    on the number of regions to stimulate, since global dynamics can quickly
    become complex. Additionally, to demonstrate the many scenarios that can
    be set up in TVB, we simulated the same brain network model under the
    influence of a stimulus, first without noise }(Fig. \ref{Fig11:sampenAB}A:
    using Heun deterministic method) and then with white noise (Fig.
    \ref{Fig11:sampenAB}B: using Heun stochastic method). \add{The first approach
    makes it easier to see the perturbations induced by the stimulus and the
    propagation of activity from one region to the other. The second approach
    is a more realistic representation of the neural activity.}
    
    \add{Results of the proposed modeling protocol are presented in} Fig. 
    \ref{Fig12:sampen_res}\add{ where the EEG traces from channel Oz for 
    the resting and evoked states are shown together with the MSE estimates.} 

    \add{Scripts to reproduce results from} Figs.\ref{Fig11:sampenAB} and 
    \ref{Fig12:sampen_res} \add{are available in the distribution packages of 
    TVB.}
    
    With the availability of surface-based simulations the challenge of 
    replicating topographic maps of different sensory systems, such as those 
    found in the primary visual cortex \citep{Hinds_2009}, could be addressed. 

    
    \paragraph*{Dynamic modeling} From both the shell and web interface it is
    possible to exploit another feature of TVB: namely, simulation continuation,
    i.e., a simulation can be stopped allowing users to modify model parameters,
    scaling factors, apply or remove stimulation or spatial constraints (e.g. local
    connectivity), or make any other change that does not alter the spatiotemporal
    domain of the system or its output (integration step, transmission speed and
    spatial support) and then resumed without the need of creating a new
    \tvbclass{Simulator} instance. Furthermore, this capability opens the
    possibility to dynamically update the simulation at runtime. Such a dynamic
    approach leads toward an adaptive modeling scheme where stimuli and other
    factors may be regulated by the ongoing activity (this last feature can be handled
    only from the scripting interface for the moment).
     
%##--------------------------------------------------------------------------##%
%##                               DISCUSSION                                 ##%
%##--------------------------------------------------------------------------##%

    \section{Discussion} 
    % The bombs and stuff 

    We have presented the architecture and usage of TVB, a neuroinformatics
    platform developed for simulations of network models of the full brain. Its
    scientific core has been  developed by integrating concepts from theoretical,
    computational, cognitive and clinical neuroscience, with the aim to integrate
    neuroimage modalities along with the interacting mesoscopic and macroscopic
    scales of a biophysical model of the brain. From a computational modeling
    perspective  TVB constitutes an alternative to approaches such as the work of
    \citet{Riera_2005} and  more recently that of \citet{Valdes-Sosa_2009}, as well
    as other relevant studies mentioned in the main text of this article. From a
    neuroinformatics perspective, TVB lays the groundwork for the integration of
    existing paradigms in the theory of large-scale models of the brain, by
    providing a general and flexible framework where the advantages and limitations
    of each approach may be determined. It also provides the community with a
    technology, that until now had not been publicly available, accessible by
    researchers with different levels and backgrounds, enabling systematic
    implementation and comparison of neural mass and neural field  models,
    incorporating biologically realistic connectivity and cortical geometry and with
    the potential to become a novel tool for clinical interventions. While many
    other environments simulate     neural activity at the level of neurons (Brian
    simulator, MOOSE, PCSIM,     NEURON, NEST, GENESIS) \citep{Goodman_2008,
    Brette_2011,Ray_2008,Hines_2001,Gewaltig_2007scholarpedia,Pecevski_2009},  even mimicking a
    number of specific brain functions \citep{Eliasmith_2012}, they, most
    importantly, do not consider the space-time structure of full brain connectivity
    constraining whole brain neurodynamics, \add{as a crucial component in their
    modeling paradigm}.  \add{Other     approaches to multi-modal integration such
    as Statistical Parametric Mapping     (SPM) perform statistical fitting to
    experimental data at the level of a     small set of nodes}
    \citep{Friston_1995, Friston_2003, David_2006, Pinotsis_2011} (i.e., they are
    data-driven as in \citet{Freestone_2011}), \add{thus diverging from our approach
    that could be     categorized as a  purely ``computational neural modeling''
    paradigm as     described in}  \citet{Bojak_2011}. \add{From this perspective,
    the goal is     to capture and reproduce  whole brain dynamics by building a
    network     constrained by its structural  large-scale connectivity and
    mesoscopic     models  governing the nodes intrinsic dynamics.}
    
    % NOTE: mainly because  they do not aim to investigate large-scale models
    Also, the extension of neuronal level modeling to large brain structures
    requires vast supercomputers to emulate the large number of complex
    functional units. \change{Focusing on the brain's large-scale
    architecture, in addition to the dimension reduction accomplished through
    the mean field methods applied on the mesoscopic scale, allows TVB to make
    computer simulations on the full brain scale on workstations and network
    workstation parallel clusters, with no need to use supercomputing
    resources.} {Focusing on the brain's large-scale architecture, in addition
    to the dimension reduction accomplished through the mean field methods
    applied on the mesoscopic scale, TVB allows for computer simulations on
    the full brain scale on workstations and small computing clusters, with no
    need to use supercomputing resources.}
    
    % the large-scale structure approach
    The simulator component of TVB has the goal of simulating mesoscopic
    neural dynamics on large-scale brain networks. It does not intend to
    build brain models at the level of neurons \remove{(Goodman and Brette,
    2008, 2009; Cornelis et al., 2012)} \citep{Goodman_2009,  Cornelis_2012},
    however, it does leverage information  from microscopic models to add
    detail and enhance the performance of  the neural population models, which
    act as building blocks and functional units of the network. TVB thus
    represents a unique tool to systematically  investigate the dynamics of
    the brain, emphasizing its large-scale network  nature and moving away
    from the study of isolated regional responses, thereby considering the
    function of each region in terms of the interplay among brain regions. The
    primary spatial support (neuroanatomical data)  on top of which the large-scale 
    network model is built has a number of  implications:
    
    \begin{enumerate}

        \item  it constraints the type of network dynamics; dynamics that 
               could be further related to physiology and behavior 
               \citep{Senden_2012}; 

        \item  it permits a systematic investigation of the consequences of the 
               particular restrictions imposed by that large-scale structure 
               and the effect of changes to it. 

        \item  it provides a reliable and geometrically accurate model of 
               sources of neural activity, enabling realistic forward solutions
               to EEG/MEG based on implementations of boundary element methods 
               (BEM) or other approaches such as finite difference time domain 
               methods (FDTD).
    \end{enumerate}
    
    % the stimulation paradigm
     On the basis of the literature, theoretical and clinical studies seeking to 
     better understand and describe certain brain functions and structure use 
     stimulation as an essential part of their protocols. Stimulation is a way 
     to probe how the system respond under external perturbations adapting itself 
     to the new environmental conditions or to categorize responses when 
     stimulation represents real-life (visual, auditory, motor) sensory inputs. 
     Among the current features of TVB, the easy generation of a variety of 
     stimulation patterns is to be recognized as one of its great advantages and 
     contributions to experimental protocol design. TVB permits the development 
     of simple stimulation routines, allowing evaluation of the viability and usefulness 
     of certain stimulation procedures. 
     
     % The good science promese
     TVB represents a powerful research platform, combining experimental design
     and numerical simulations into a collaborative framework that allows 
     sharing of results and the integration of data from other applications. 
     Naturally, this leads to the potential for an increased level of 
     interaction among researchers of the broad neuroscience community. In the 
     same direction, TVB is also an extensible validation platform since it 
     supports the creation of basic modeling refinement loops, making model 
     exploration and validation a relatively automated procedure. For instance,  
     after generating a brain network model, exploring the system's parameter 
     space by adjusting parameters of both the local dynamics and the large
     scale structure can be achieved with ease. Further, effects of local 
     dynamics and network structure can be disentangled by evaluating distinct
     local dynamic models on the same structure or the same local dynamic 
     model coupled through distinct structures. This constrained flexibility
     makes it easy for modelers to test new approaches, directly compare them
     with existing approaches and reproduce their own and other researchers' 
     results. Reproducibility is indeed a required feature to validate and 
     consequently increase the reliability of scientific work 
     \citep{Donoho_2010} and the extensibility of TVB's scientific 
     components, granted by its modular design, provides a mechanism to help 
     researchers achieve this. 
     
     % The clinical dreams
     The brain network models of TVB, being built on explicit anatomical 
     structure, enable modeling investigations of practical clinical interest.
     Specifically, whenever a dysfunction or disease expresses itself as a 
     change to the large scale network structure, for instance, in the case of 
     lesions in white-matter pathways, the direct replication of this 
     structural change in TVB's brain network models is straight forward.
     
    %##--------------------------------------------------------------------------##%
    %##                               FUTURE                                     ##%
    %##--------------------------------------------------------------------------##% 
    
    \section*{Future Work}

    \add{Regarding performance, of special importance will be to evaluate all
    the parameters that have an effect on both memory usage and execution time
    for surface-based simulations. The reason is that realistic brain network
    models are built on top of surface meshes constructed by thousands of
    vertices per hemisphere ($2^{13}$ for the TVB demonstration cortical surface) but
    can easily have more than 40,000. 

    Equally important is to develop more
    tests to generally evaluate the simulation engine, paying close attention to
    keep the consistency and stability of the algorithms currently implemented.
    
    Another aspect that deserves careful attention is the description of our
    modeling approach that was largely beyond the scope of this text.
    Therefore, the theory underlying the different methods involved in the
    development of a generalized framework for brain network models is to be
    presented in future scientific publications.}

    \remove{The \tvbmodule{simulator} and \tvbdatatype{datatypes} of TVB are written in
    Python. Simulations benefit from vectorized numerical computations with NumPy
    arrays and are enhanced by the use of the \emph{numexpr} package. Although 
    this allows rather efficient single simulations, the desire to systematically
    explore the parameter spaces of the neural dynamic models, and to compare many
    connectivity matrices, has lead to the implementation of code generation 
    mechanisms for the majority of the simulator core -- producing C code for both
    native CPU and also graphics processing units (GPU), leading to a significant 
    speed up of parameter sweeps and parallel simulations (5x from Python to C,
    40x from C to GPU). Such graphics units have become popular in scientific
    computing for their relatively low price and high computing power. Going 
    forward, the GPU implementation of TVB will require testing and optimization
    before placing it in the hands of users. }

    % accessible to the public
    \remove{TVB is meant to be a large collaborative project, available
    not only under the form of customized Python modules but as an
    open-source cross-platform toolbox. We therefore see the need of
    creating a public repository, this time using the distributed version
    control system \emph{git} (Chacon, 2009), to make accessible
    the scientific core and to gather, manage and integrate contributions from
    the community. Special effort is being made to provide a good code-coverage 
    including regression tests. The public git repository is scheduled for March
    2013 along with the latest TVB release.} 
 
    \remove{Another goal of great relevance in clinical neuroscience research, is to
    implement a new Monitor based on a forward model of neural activity to  
    stereoelectroencephalography (sEEG). This invasive technique is used to
    identify areas of the human brain, which are the source of epileptic seizures.
    Each patient has a different implantation scheme of depth needle like
    electrodes. This new \tvbclass{Monitor} class would output simulated 
    measurements from virtual sensors inside the brain. }

    % the educational platform - spread the word   
    To allow a most optimal dissemination of knowledge in TVB we are currently 
    developing a web-based educational platform that will allow training on the
    usage of TVB, as well as serve as a key reference.
   
    % new datasets
    As simulations in TVB are built on the large-scale anatomical structure of
    the human brain, continued work to integrate new, reliable, sources of 
    structural data is essential to the progress of the platform. An obvious
    future resource in this regard will be the newly developed database of the 
    Human Connectome Project \citep{Essen_2012b, Essen_2012c}.
    
    \section*{Information Sharing Statement (License)}
    The data and software in this study belong to an ongoing project; \add{it 
    is free software and licensed under the GNU General Public License version 2 
    as published by the Free Software Foundation.} The latest releases of \TVB 
    including the source code and demo data are free to download from 
    \url{http://www.thevirtualbrain.org}. The source code \add{available in the
    public repository} includes the latest experimental features regarding GPU 
    implementation. 
    
    %\section*{Disclosure/Conflict-of-Interest Statement}

    \section*{Acknowledgments}
    \paragraph{Funding\textcolon} The research reported herein was supported by the Brain 
    Network Recovery Group through the James S. McDonnell Foundation and the FP7-ICT
    BrainScales. PSL is supported by a doctoral fellowship from Ministere de la Recherche.

        %include bibliography and its style 
        \bibliographystyle{apalike}
        \bibliography{../tvbphd/contents/Bibliography/TVB-Bibliography}
%-------------------------------------------------------------------------------%
    \section*{}
    
    \begin{figure*}
        \centering
        %\includegraphics[width=\textwidth]{FIG01_tvb_architecture.jpg}
        \caption{\TVB Architecture: TVB provides two independent interfaces 
        depending on the interaction with users. Blocks in the back-end are 
        transparently used by different top application layers. 
        TVB-\tvbdatatype{Datatypes}, are the common language between different 
        components (analyzers, visualizers, simulator, uploaders).\add{ They represent ``active data'' 
        in the sense that, when TVB is configured with
        a database, data contained in TVB-\tvbdatatype{Datatypes} instances are 
        automatically persistent}. Currently the console interface works without
        the storage layer, keeping the results just in memory. S-Users need to 
        manually handle data import and export operations.}
      \label{Fig01:tvb_soft_architecture}
  \end{figure*}

  \begin{figure*}
      %\caption{FIG 2}
       \centering
       %\includegraphics{FIG02_ui_main_areas_map.jpg}
       \caption{\change{Typical stages of a workflow through \TVB web 
       interface: USERS configure their personal information and technical 
       settings for TVB's UI. They can create and manage several PROJECTS and
       interact with the SIMULATOR, where simulations are configured and 
       launched. Results can be immediately analyzed and visualized to have a
       quick overview of the ongoing project. Later on, simulated data can be
       exported in HDF5 format and may be used outside of the framework. In 
       the SIMULATOR area a history of launched simulations is kept to have 
       the traceability of any modifications that took place in the simulation
       chain.}{Main working areas of \TVB's web interface: in \textbf{USER} 
       personal information (account settings) as well as hardware and 
       software preferences (technical settings) are configured. Through the
       \textbf{PROJECT} area users access and organize their projects, data,
       figures and the operations dashboard. Input and output simulated data
       can be exported in HDF5 format and may be used outside of the framework.
       Brain network models and execution of simulations are configured and 
       launched respectively in \textbf{SIMULATOR}. In this area results can 
       be immediately analyzed and visualized to have a quick overview of the current model. 
       A history of launched simulations is kept to have the 
       traceability of any modifications that took place in the simulation 
       chain. \textbf{STIMULUS} provides a collection of tools to build 
       stimulation patterns that will be available to use in the simulations.
       Finally, \textbf{CONNECTIVITY} provides an interactive environment to 
       the edit and visualize connectivity matrices.}
       }
       \label{Fig02:tvb_wui}
   \end{figure*}

   \begin{figure*}%
    %\caption{FIG 3}
    \centering
    %\includegraphics[width=\textwidth]{FIG03_ui}
    \caption{\add{UI screenshots}. \textbf{(A)} \textbf{SIMULATOR} Area. 
    Having multiple panels allows a quick overview of previous simulations 
    (left), model parameters for the currently selected simulation (middle), 
    and summary displays of the data associated with the currently selected 
    simulation (right). \textbf{(B)} shows the interface for editing and 
    visualising the structural connectivity, for one of the six possible 
    connectivity visualisations. \textbf{(C)} \textbf{PROJECT} Area - 
    operations \change{tab}{dashboard}. \remove{This dashboard lists all the
    operations with their current status (running, error, finished), as well 
    as who created the operation, wall-time, execution time and icons for 
    resulting DataTypes.} On the left column, users can compose filters to 
    search through all the operations on the list.}%
  \label{Fig03:ui_screenshots}
  \end{figure*}

  \begin{figure*}%
  %\caption{FIG 4}
  \centering
  %\includegraphics[width=\textwidth]{FIG04_spatial_support}
  \caption[Dtypes]{Demonstration datasets exist in TVB for the anatomical 
            structure on which simulations are built, including a triangular
            mesh surface representation of the neocortex \textbf{(A)} and 
            white matter fiber lengths \textbf{(B)}. However, new data from 
            structural imaging such as MRI, DTI, and DSI for individual 
            subjects, as well as data from the literature can be used and 
            wrapped in a TVB-\tvbdatatype{Datatype}.
         }
     \label{Fig04:connectivity}%
 \end{figure*}

 \begin{figure*}
    %\caption{FIG 5}
    \centering
    %\includegraphics[width=\textwidth]{FIG05_scientific_library}
     \caption{Diagram of the configurable elements for building a brain 
       network model and launching a simulation. TVB can
       incorporate cortical connectivity information from an individual's 
       tractographic and cortical geometry data. The \tvbdatatype{Connectivity}
       object contains matrices defining the connection strengths and time 
       delays via finite signal transmission speed between all regions, while the folded 
       \tvbdatatype{Cortical Surface} mesh provides the spatial support for
       finer resolution models. In the latter case a 
       \tvbdatatype{Local Coupling} defines the interaction between 
       neighboring nodes. In its simplest form local connectivity is spatially
       invariant, however, support exists for spatial inhomogeneity. Signal 
       propagation via local connectivity is instantaneous (no time delays),
       which is a reasonable approximation considering the short distances 
       involved.
       \textbf{Together, the cortical surface with its local connectivity, the 
       long-range connectivity matrix, and the neural mass models defining the
       local dynamics define a full brain network model}. Additionally,
       stimulation can be applied to a simulation. The stimulation patterns 
       are built in terms of spatial and temporal equations chosen 
       independently. For region-based network models, it is only possible to
       build time dependent stimuli since there is not a spatial extent for a
       region node. However, node-specific weightings can be set to modulate
       the intensity of the stimulus applied to each node. For surface-based
       models, equations with finite spatial support are evaluated as a 
       function of distance from one or more focal points (vertices of the 
       surface), where the equation defines the spatial profile of the 
       stimuli. The neural source activity from both region or surface-based
       approaches can be projected into EEG, MEG and BOLD \citep{Buxton_1997,
       Friston_2000} space using a forward model \citep{Breakspear_2007}.
     }
      \label{Fig05:tvb_sim}
  \end{figure*}

  \begin{figure*}[htp]%
  %\caption{FIG 6}
  \centering
  %\includegraphics{FIG06_vis_anz}
  \caption[Visualizers]{Visualizers. \textbf{(A)} Histogram of a graph metric
  as a function of nodes in the connectivity matrix. \textbf{(B)} A 2D 
  projection of the head. The color map represents a graph metric computed on
  the connectivity matrix. \textbf{(C)} EEG visualizer combines a rendered 
  head surface, an overlay with the sensors positions and an interactive 
  time-series display. \textbf{(D)} An animated display of the spatiotemporal
  pattern applied to the cortical surface. Red spots represent the focal 
  points of the spatial component of the stimulus.
     }%
  \label{Fig06:ui_vis}%
  \end{figure*}

  \begin{figure*}%   
  %\caption{FIG 7}   \centering
  %\includegraphics[]{FIG07_}            
    \caption[Results]{\textbf{(A)} \add{As
      expected for fixed time-step schemes,} execution times scale linearly with
      the number of integration steps. We used seven values of simulation
      lengths (1, 2, 4, 8, 16, 32 and 64 seconds) and five values of integration
      time step: $dt=0.25=2^{-2}$, $dt=0.125=2^{-3}$, $dt=0.0625=2^{-4}$,
      $dt=0.03125=2^{-5}$ and $dt=0.015625=2^{-6}$ milliseconds). For each
      possible combination 100 simulations were performed. The network model
      consisted of 74 nodes (with 2 state variables and one mode per node).
      Numerical integration was based on Heun's stochastic method. We plot the
      average execution time with the error bars representing the standard
      deviation over simulations. The inset shows a narrower range for
      simulation lengths between 1 and 4 seconds. Axes units and color code are
      the same as those displayed in the main plot. \textbf{(B)} Here, execution
      times are shown as a function of the integration time step size, $dt$, for
      two different number of nodes  (solid and dashed lines correspond to connectivity matrices of 64
      and 128 nodes respectively) for a specific conduction speed (4mm/ms) and simulation length (64 seconds).
      Both axes are in logarithmic scale with base 2. In this case, halving  
      $dt$ or doubling the number of nodes in the connectivity matrix, $N$, 
      doubles the running time. However, as mentioned in the text, for larger networks
      execution times seem to grow quadratically as a function of the number of nodes in the network.
      Further tests need to be developed to understand this behavior.
      }
      \label{Fig07:ets}%   
  \end{figure*}

  \begin{figure*}
  %\caption{FIG 8}
  \centering
  %\includegraphics[width=0.5\textwidth]{FIG08_pp_interactive}
       \caption{Phase portrait using TVB's interactive phase plane tool 
              (accessible from both shell and graphical interfaces): the
               blue line corresponds to a trajectory of a single oscillator
               node isolated and without noise, \change{integration scheme 4th
               Runge-Kutta}{4th order Runge-Kutta integration scheme}. In the 
               bottom panel, the corresponding trajectories of both the $v(t)$
               and $w(t)$ state variables of the model are shown. The activity
               exhibits oscillations at approximately 40 Hz.}
   \label{Fig08:results_reproduce_deco_pp}
   \end{figure*}

   \begin{figure*}%
  %\caption{FIG 9}
  \centering
  %\includegraphics{FIG09_replicate}
        \caption[Results]{
        \textbf{(A)} the activity of individual regions are illustrated in 
        colored lines. The black line represents the average activity over the
        network nodes. Here brain regions are weakly coupled changing both the
        collective and local dynamics of the network. \textbf{(B)} Using TVB 
        scientific library as a python module we can conveniently run thousands
        of simulations in parallel on a cluster. \add{Note that TVB 
        parallelizes different tasks e.g. simulations and analyses, taking 
        advantage of multi-core systems, however it does not parallelize the 
        processes themselves.} \change{This allows}{Simultaneous simulations 
        allow for} a systematic parameter space exploration to rapidly gain 
        insights of the whole brain dynamics repertoire. In this plot, the 
        magnitude and color scale correspond to one the variance computed over
        all the elements of the N-dimensional output array 
        (\emph{Global Variance}). Simulations were performed on a cluster based
        on the Python demo scripts available in the release packages. On of the
        major strengths of \TVB is that G-Users are enabled to launch parameter
        sweeps through the UI without the need to know \change{parallel 
        programming}{how to submit parallel jobs} (see Fig. \ref{Fig10:pse_ui}).
          }
      \label{Fig09:results}%
  \end{figure*}


    \begin{figure*}
        %\caption{FIG 10}
        \centering
        %\includegraphics{FIG10_pse_UI}
        \caption{One of TVB's major strengths is the capability to launch 
        parallel simulations through the UI. We show a screenshot of the the 
        resulting display when sweeping across two different parameters of the
        \emph{Generic2dOscillator} model. Here each data point represents two 
        metrics: size is mapping the \emph{Global Variance} and color 
        corresponds to the \emph{Variance of the nodes Variance}. These results
        provide a topography of the stability space allowing users to 
        distinguish, and thus select, combinations of critical parameters 
        values.}
      \label{Fig10:pse_ui}
  \end{figure*}

    \begin{figure*}%
        %\caption{FIG 11}
        \centering
        %\includegraphics{FIG11_ventral_stream}
        \caption{
        \textbf{(A)} The upper left blue panel shows the raw traces of nodes V2
        and V1; the latter stimulated with a rectangular pulse of width equal 
        to 5 milliseconds \add{and repetition frequency of 1 Hz.} Signals are 
        normalized by their corresponding maximum value. The right blue panel 
        show the signals for a shorter period of time. Amplitudes are not 
        normalized to emphasize the relative difference between the two 
        regions. Middle panels illustrate the stimulus pattern. Lower red 
        panels display the activity as projected onto EEG space and recorded 
        from channels Oz and O1. The default EEG cap in TVB consists of 62 
        scalp electrodes distributed according to the 10-20 international 
        system \citep{Klem_1999}. \add{In this simulation a deterministic 
        integration scheme was employed to obtain the time-series of neural 
        activity, since noise was not applied to the model's equations}. 
        \change{The same description as in  \textbf{(A)} applies. (B) However,
        notice the slow dumped oscillations after stimulus onset. The 
        approximate characteristic frequency of these waves is 10 Hz.}{\textbf{(B)} 
        The same description as in \textbf{(A)} applies. The main difference 
        with the previous simulation is that here white noise was added to the 
        system.}    
     }
      \label{Fig11:sampenAB}%
  \end{figure*}


\begin{figure*}
        %\caption{FIG 12}
        \centering
        %\includegraphics[width=\textwidth]{FIG12_mse.jpg}
        \caption{The green and blue panels show EEG recordings from electrode 
        Oz during the resting state , i.e., in the absence of stimulation and 
        in the stimulated condition respectively, notice the slow damped 
        oscillations after stimulus onset at a approximately 10 Hz; the light 
        gray trace depicts the stimulation pattern. The bottom panel displays 
        \change{MSE}{multiscale entropy estimates} computed on the Oz 
        time-series at different temporal scales \change{.}{using the dataset
        obtained by means of a stochastic integration scheme.}}
       \label{Fig12:sampen_res}
\end{figure*}
%------------------------------------------------------------------------------
% Pretty model template
%\newpage
%\pagebreak
\section*{}
%\include{sup_model_template}

\end{document}
